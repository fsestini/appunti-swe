\newglossaryentry{Ingegneria del Software}
{
	name={Ingegneria del Software},
	description={Approccio sistematico, disciplinato e quantificabile allo sviluppo, il mantenimento e il ritiro del software}
}

\newglossaryentry{best practice}
{
	name={best practice},
	description={Metodo di lavoro che l'esperienza e lo studio hanno provato avere i migliori risultati in circostanze specifiche e ben note}
}

\newglossaryentry{efficienza}
{
	name=efficienza,
	description={Valore indicativo inversamente proporzionale alla quantità di risorse consumate nell'esecuzione di un processo di produzione}
}

\newglossaryentry{efficacia}
{
	name=efficacia,
	description={Valore indicativo di quanto un prodotto o un processo soddisfano i requisiti di conformità e qualità}
}

\newglossaryentry{ciclo di vita}
{
	name={ciclo di vita},
	description={Insieme di tutti gli stati assunti dal prodotto dal concepimento al ritiro, e dei processi che lo portano da uno stato all'altro}
}

\newglossaryentry{stakeholder}
{
	name=stakeholder,
	description={Soggetto (o un gruppo di soggetti) influente nei confronti di un'iniziativa economica, sia essa un'azienda o un singolo progetto}
}

\newglossaryentry{processo}
{
	name={processo},
	description={Struttura metodologica, normativa e strategica che caratterizza, suddivide e ordina le attività di progetto}
}

\newglossaryentry{milestone}
{
	name={milestone},
	description={Stato del progetto in cui e' possibile accertare un progresso significativo}
}
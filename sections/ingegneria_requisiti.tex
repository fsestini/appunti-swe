\section{Ingegneria dei requisiti}

I requisiti di un sistema sono descrizioni delle funzionalità di un sistema, i
servizi che fornisce e i vincoli sulle sue operazioni. In particolare, secondo
il glossario IEEE, un requisito è:

\begin{enumerate}
  \item Una condizione (\frgnword{capability}) necessaria a un utente per
        risolvere un problema o raggiungere un obbiettivo (visione dal lato del
        bisogno);
  \item Una condizione (\frgnword{capability}) che deve essere soddisfatta o
        posseduta da un sistema per adempiere a un obbligo (contratto, standard,
        specifica, documento formale) (visione dal lato della soluzione);
  \item Una descrizione documentata di una condizione (\frgnword{capability})
        come in 1. e 2.
\end{enumerate}

\todo{slide 3/42 si riferisce alla verifica e alla validazione. inserirla nelle
sezioni relative?}

\todo{slide 4..6/42 ??}

L'attività durante il quale si delineano, analizzano, documentano e verificano i
servizi, le funzionalità e i vincoli di un sistema è chiamato \strong{ingegneria
dei requisiti}. L'ingegneria dei requisiti è parte integrante dell'ingegneria di
sistema e attività chiave del processo di sviluppo, e richiede competenze
specializzate. Rappresenta l'insieme di attività necessarie al trattamento
sistematico dei requisiti.

\paragraph{Attività}

\begin{itemize}
  \item Analisi
    \begin{itemize}
      \item Analisi dei bisogni (identificare cosa so o leggo che devo fare) e
            delle fonti (luoghi dal quale scaturiscono altri bisogni)
      \item Classificazione dei requisiti: non si vuole avere un insieme
            disorganizzato di requisiti, ma organizzarli in modo strutturato e
            ordinato.
      \item Modellazione concettuale del sistema: diagramma Casi d'Uso, che
            rappresenta il punto di vista dell'attore (qualsiasi entità esterna
            al sistema che interagisce con esso). Definisce i bisogni esterni
            del sistema.
      \item Assegnazione dei requisiti a parti distinte del sistema
      \item Negoziazione con il committente e con i sotto-fornitori: i requisiti
            devono essere concordati e negoziati con gli \frgnword{stakeholder},
            in modo che vengano fissati quelli irrinunciabili e individuati
            quelli secondari che, in mancanza di risorse, potranno essere
            scartati. Il confronto con le fonti (committente, utenti, ecc.) è
            fondamentale.
    \end{itemize}

    Durante l'attività di analisi dei requisiti è necessario equipaggiarsi, per
    metodo e per strumenti, per fare verifica e validazione.

  \item Specifica di verifica e validazione
    \begin{itemize}
      \item Predisposizione di revisione interna/esterna
      \item Predisposizione di prove e dimostrazioni
    \end{itemize}
  \item Analisi dei bisogni e delle fonti
    \begin{itemize}
      \item Propedeutica a identificazione, analisi, specifica e classificazione
            dei requisiti
    \end{itemize}
  \item Modellazione concettuale del sistema
    \begin{itemize}
      \item Partizionamento in componenti a scopo di allocazione dei requisiti
            (come nel diagramma dei casi d'uso)
      \item Non è progettazione della soluzione
    \end{itemize}
  \item Attribuzione dei requisiti a `parti'
\end{itemize}

L'analisi dei requisiti è conclusa dalla definizione di una \strong{\frgnword{
requirements baseline}}. Una baseline è un aggregato di \frgnword{configuration
item} associato a un punto di sviluppo strategico, chiamato
\frgnword{milestone}. Si raggiunge una \frgnword{requirements baseline} quando
vengono definiti tutti i requisiti che importano. La definizione della
\frgnword{requisiti baseline} costituisce il punto di fine dell'analisi dei
requisiti e di inizio della fase di progettazione.

\paragraph{Processi coinvolti}

\begin{itemize}
  \item Documentazione: serve uno strumento che renda i requisiti oggettivi,
        discutibili (oggetto plausibile di discussione) e non ambigui (uso di
        diagrammi, evitando il linguaggio naturale). La poca narrazione
        associata ai diagrammi deve essere di supporto, tecnica e sintetica, e
        appoggiarsi a un glossario per eliminare l'ambiguità dei termini.

    \begin{itemize}
      \item Studio di fattibilità
      \item Analisi dei requisiti
    \end{itemize}

  \item Gestione e manutenzione dei prodotti: i requisiti cambiano nel tempo per
        aggiunta, rimozione e estensione. Serve un insieme di regole, procedure
        e strumenti per gestire questi cambiamenti.

    \begin{itemize}
      \item Tracciamento dei requisiti: permette di sapere, per ogni passo di
            avanzamento nel progetto, a quale requisito risponde. Rappresentando
            il piano di attività come un grafo direzionato e aciclico, il
            tracciamento comunica il motivo di ciascun arco.
      \item Impostazione e gestione della configurazione
      \item Gestione dei cambiamenti
    \end{itemize}

\end{itemize}

\subsection{Studio di fattibilità}

Lo studio di fattibilità precede l'analisi dei requisiti, e si occupa di
valutare la fattibilità del progetto in termini di rischi, costi e benefici, e
decidere se sia conveniente procedere con il lavoro.

\begin{itemize}
  \item Fattibilità tecnico-organizzative: valutare se si hanno le competenze e
        gli strumenti necessari, le soluzioni algoritmiche e architetturali e le
        piattaforme idonee per l'esecuzione;
  \item Rapporto costi/benefici: valutare il rapporto tra il costo di produzione
        e la redditività dell'investimento, nei confronti del mercato attuale e
        futuro;
  \item Individuazione dei rischi: capire le complessità e le incertezze;
  \item Valutazione delle scadenze temporali;
  \item Valutazione delle alternative

    \begin{itemize}
      \item Scelte architetturali: sistema centralizzato o distribuito, modello
            client-server, ecc.
      \item Strategie realizzative: riuso o sviluppo ex-novo
      \item Strategie operative
    \end{itemize}

\end{itemize}

Lo studio di fattibilità è un'attività preliminare che non può impiegare troppo
tempo; è quindi un processo rapido, i cui risultati però non vengono scartati,
ma costituiscono una base di partenza per l'analisi dei requisiti.

\subsection{Tecniche di analisi}

\begin{itemize}
  \item Analisi dei bisogni e delle fonti

    \begin{itemize}
      \item Studio del dominio, con osservazione dei comportamenti dell'utente
            finale e dell'ambiente d'uso;
      \item Interazione con il cliente: instaurare un rapporto con il cliente
            non invadente, intelligente e strutturato, definito intervista. Le
            interviste costituiscono un dialogo da cui esce un resoconto
            approvato da entrambe le parti;
      \item Discussioni creative (brainstorming), coinvolgono tre figure

        \begin{itemize}
          \item Un gruppo di persone che discuta il problema
          \item Un `facilitatore', che aiuti la discussione a convergere in un
                tempo finito, che fissi un tempo e aiuti a valutare il risultato
                della discussione. Il facilitatore non può essere
                \frgnword{biased} e non è coinvolto direttamente nella
                discussione
          \item Una persona che tenga le minute della discussione, che ne
                registri i punti salienti
        \end{itemize}

      \item Prototipazione: può essere interna, a vantaggio del fornitore, o
            esterna, per arricchire il rapporto con il cliente; può essere
            usa-e-getta o incrementale;
    \end{itemize}

  \item Dominio, campo di applicazione del prodotto: a quali bisogni risponde, e
        quali problematiche coinvolge;
  \item Acquisizione delle competenze

    \begin{itemize}
      \item Documentazione preesistente
      \item Interviste agli utenti potenziali
      \item Studio delle soluzioni esistenti
    \end{itemize}

  \item Glossario

    \begin{itemize}
      \item Raccoglie e definisce i termini chiave del dominio
      \item Da sottoporre alla verifica e approvazione del committente
      \item Consolidato mediante uso nelle interviste
    \end{itemize}

\end{itemize}

\subsection{Classificazione dei requisiti}

E' fondamentale organizzare i requisiti in maniera strutturata e organizzata.
Suddividere i requisiti in diversi livelli di dettaglio è utile per comunicare
informazioni sul sistema a differenti tipologie di lettori.

Si può individuare una distinzione tra \strong{requisiti utente} e
\strong{requisiti di sistema}:

\begin{itemize}
  \item Requisiti utente: affermazioni, in forma di linguaggio naturale e
        diagrammi, di quali servizi il sistema dovrebbe offrire all'utente, e i
        vincoli secondo il quale dovrebbe operare;
  \item Requisiti di sistema: descrizioni dettagliate delle funzioni, servizi e
        vincoli operativi del sistema. Costituiscono una documentazione esatta di
        cosa deve essere implementato.
\end{itemize}

I requisiti descrivono \strong{attributi di prodotto} o \strong{attributi di
processo}. Gli attributi di prodotto definiscono le caratteristiche del sistema;
rispondono alla domanda `cosa devo fare?'. Gli attributi di processo pongono
vincoli sui processi impiegati nel progetto (linguaggio di programmazione da
usare, strumento, metodo, ecc.); rispondono alla domanda `come devo farlo?'.

Gli attributi di prodotto esprimono

\begin{itemize}
  \item requisiti funzionali, che determinano le capacità di calcolo richieste
        al sistema (\frgnword{capabilities});
  \item requisiti prestazionali;
  \item requisiti qualitativi;
\end{itemize}

% TODO espandere.
Gli attributi di processo esprimono ulteriori requisiti extra-funzionali, che
riducono i gradi di libertà disponibili nella definizione della soluzione.

I requisiti devono essere verificabili: chi impone un requisito deve avere idea
di come accertarne il soddisfacimento; chi è chiamato a soddisfare un requisito
deve saperne valutare costo e complessità di verifica.

\begin{figure}[h!]
  \centering
  \begin{tabular}{|l|l|}
    \hline
    \strong{Tipologia di requisito} & \strong{Modalità di verifica} \\
    \hline
    Requisiti funzionali & test, dimostrazione formale, revisione \\
    \hline
    Requisiti prestazionali & misurazione \\
    \hline
    Requisiti qualitativi & verifica \emph{ad hoc} \\
    \hline
    Requisiti dichiarativi & revisione \\
    \hline
  \end{tabular}
\end{figure}

I requisiti hanno diversa utilità strategica:

\begin{itemize}
  \item Obbligatori: irrinunciabili per un qualsiasi \emph{stakeholder};
  \item Desiderabili: non strettamente necessari ma a valore aggiunto
        riconoscibile;
  \item Opzionali: relativamente utili oppure contrattabili in seguito;
\end{itemize}

I requisiti non devono essere in conflitto tra loro.

\subsection{Attività di analisi}

L'attività di analisi si svolge in diverse fasi:

\begin{enumerate}
  \item Studiare e definire il problema da risolvere: identificare il prodotto
        da commissionare (cliente), capire cosa deve essere realizzato (cliente
        e fornitore) e definire gli accordi contrattuali (cliente e fornitore).
  \item Verificare le implicazione di conto e di qualità: la soddisfazione del
        cliente è relativa ai requisiti.
  \item Accertare la soddisfacibilità dei requisiti rispetto ai vincoli di
        processo.
  \item I requisiti devono essere tutti e soli quelli necessari e sufficienti:
        nessun bisogno trascurato (\strong{chiusura}), nessuna caratteristica
        superflua (\strong{sinteticità}).
  % TODO wat?? che significa?
  \item Una priorità relativa può essere assegnata ai requisiti confermati.
\end{enumerate}

% TODO: probabilmente i seguenti contenuti vanno distribuiti nelle varie
% sezioni del documento.
\subsection{Estratti da IEEE 830-1998}

Un corretto SRS limita il \frgnword{range} di soluzioni progettuali valide,
ma non ne specifica alcuna.

Un SRS dovrebbe essere:

\begin{itemize}
  \item Corretto: ogni requisito indicato verrà soddisfatto dal prodotto;
  \item Non ambiguo: ogni requisito ha un'unica interpretazione. Questo implica
        che ogni caratteristica del prodotto finale sia descritta da un unico
        singolo termine dal significato non ambiguo e indipendente dal contesto;
  \item Completo: un SRS è completo se e solo se include i seguenti elementi:

    \begin{itemize}
      \item Tutti i requisiti significativi;
      \item Definizioni delle risposte del software a ogni possibile classe di
            input in ogni possibile situazione, sia valido che non valido;
      \item Didascalie complete per figure, tabelle e diagrammi, e definizioni
      di tutti i termini e unità di misura utilizzate.
    \end{itemize}

  \item Consistente: consistenza con altri documenti di progetto;
  \item Ordinato per importanza e/o stabilità: ogni requisito deve essere
        identificato in modo da rendere chiare le differenze in termini di
        importanza e stabilità;
  \item Verificabile: un SRS è verificabile se ogni requisito in esso è
        verificabile. Un requisito è verificabile se e solo se esiste una
        procedura dal costo finito (possibilmente automatizzata) che permetta di
        verificare che il prodotto soddisfi il requisito (requisiti non
        verificabili spesso includono diciture come "funziona bene" oppure "
        solitamente dovrebbe succedere");
  \item Modificabile: la struttura e lo stile del documento sono tali che ogni
        modifica può essere fatta facilmente, completamente e consistentemente
        mantenendo la struttura intatta. Tipicamente si realizza organizzando i
        contenuti in \frgnword{table of content}, con \frgnword{cross-
        referencing} esplicito e evitando ridondanza;
  \item Tracciabilità: un SRS è tracciabile se l'origine di ogni requisito è
        chiara. Vi sono due tipi di tracciabilità:

        \begin{itemize}
          \item \frgnword{backward traceability}, verso stadi precedenti
                dello sviluppo. Si verifica se ogni requisito identifica
                esplicitamente la fonte;
          \item \frgnword{forward traceability}, verso stadi successivi di
                sviluppo. Si verifica se ogni requisito ha un nome che lo
                identifica univocamente.
        \end{itemize}

\end{itemize}

\subsection{Verifica dei requisiti}

La verifica viene eseguita su un documento organizzato, tramite
\frgnword{\strong{walkthrough}} o \strong{ispezione}. Viene usata una matrice
delle dipendenze per il tracciamento.

E' importante, ai fini della verifica, che la SRS (\frgnword{Software
Requirements Specification}) possieda le seguenti caratteristiche:

\begin{itemize}
  \item Chiarezza espositiva, evitando il più possibile il linguaggio naturale a
        favore di schemi e diagrammi;
  \item Chiarezza strutturale: separazione tra requisiti funzionali e non-
        funzionali; classificazione precisa, uniforme e accurata;
  \item Atomicità e aggregazione: requisiti elementari; correlazioni chiare ed
        espliciti;
\end{itemize}

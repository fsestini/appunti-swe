\section{Ingegneria dei requisiti}
I requisiti di un sistema sono descrizioni delle funzionalità di un sistema, i servizi che fornisce e i vincoli sulle sue operazioni. In particolare, secondo il glossario IEEE, un requisito è:

\begin{enumerate}
	\item Una condizione (\frgnword{capability}) necessaria a un utente per risolvere un problema o raggiungere un obbiettivo (visione dal lato del bisogno);
	\item Una condizione (\frgnword{capability}) che deve essere soddisfatta o posseduta da un sistema per adempiere a un obbligo (contratto, standard, specifica, documento formale) (visione dal lato della soluzione);
	\item Una descrizione documentata di una condizione (\frgnword{capability}) come in 1. e 2.
\end{enumerate}

\todo{slide 3/42 si riferisce alla verifica e alla validazione. inserirla nelle sezioni relative?}

\todo{slide 4..6/42 ??}

L'attività durante il quale si delineano, analizzano, documentano e verificano i servizi, le funzionalità e i vincoli di un sistema è chiamato \strong{ingegneria dei requisiti}. L'ingegneria dei requisiti è parte integrante dell'ingegneria di sistema e attività chiave del processo di sviluppo, e richiede competenze specializzate. Rappresenta l'insieme di attività necessarie al trattamento sistematico dei requisiti.
\section{Processi software}
Un \gls{processo} ingegneristico consiste in un insieme di attività correlate atte a trasformare uno o più elementi in \frgnword{input} in un prodotto \frgnword{output}, consumando risorse. I processi software sono ciò che caratterizza il lavoro dell'ingegnere del software. Includono una pre- ed una postcondizione e decompongono le attività in incarichi (\frgnword{task}), la più piccola unità di lavoro soggetta a gestione. Così come le procedure di un programma, le attività di un processo sono tipicamente ripetute più volte, in modo iterativo, secondo il principio del \strong{riuso}.

\todo{Controllo di processo: trattato nella parte di qualità di processo... Inserire anche qua?}

I processi software si categorizzano in \strong{primari}, di \strong{supporto} e \strong{organizzativi}.

\paragraph{Processi primari}
I processi primari sono messi in atto dalle parti primare di un progetto, quali l'acquirente, il fornitore, lo sviluppatore, l'utilizzatore e il mantenitore. Essi sono:

\begin{itemize}
	\item Acquisizione;
	\item Fornitura;
	\item Sviluppo;
	\item Gestione operativa (utilizzo);
	\item Manutenzione;
\end{itemize}

\paragraph{Processi di supporto}
Questi processi vengono impiegati ed eseguiti da altri processi quando necessario, e costituiscono supporto al processo che li esegue. Contribuiscono al successo e alla qualità del prodotto software.

\begin{itemize}
	\item Documentazione;
	\item Gestione di configurazione;
	\item \frgnword{Quality assurance};
	\item Verifica;
	\item Validazione;
	\item Analisi congiunta (\frgnword{joint review});
	\item Revisione (\frgnword{audit});
	\item Risoluzione di problemi;
\end{itemize}

\paragraph{Processi organizzativi}
I processi organizzativi sono impiegati dall'azienda nell'atto di implementare e migliorare costantemente la struttura sottostante costituita da processi e personale.

\begin{itemize}
	\item Gestione dei processi;
	\item Gestione delle infrastrutture;
	\item Miglioramento di processo;
	\item Formazione del personale;
\end{itemize}

\subsection{Processi, aziende, progetti}
Non esistono processi migliori in generale: i processi vanno selezionati, adattati e messi in atto appropriatamente in relazione al progetto, il contesto aziendale e il dominio di applicazione. Un \strong{dominio di applicazione} definisce un insieme di requisiti, terminologie e funzionalità comuni a tutti i programmi relativi ad una particolare area dello sviluppo software. Stessi domini applicativi condividono, ragionevolmente, gli stessi processi software.

\begin{itemize}
	\item Processo standard: riferimento di base, condiviso dalle aziende di uno stesso dominio applicativo;
	\item Processo definito: specializzazione di un processo standard, adattato ai particolari bisogni dell'azienda;
	\item Processo di progetto: istanziazione di un processo definito. Consuma le risorse dell'azienda per ottenere obbiettivi prefissati e limitati nel tempo (progetti);
\end{itemize}

% Process specifications \emph{do not} determine the choice of a life cycle model. Factors that instead influence the choice are
% Le specifiche dei processi non determinano la scelta del modello di ciclo di vita da attuare.
% \begin{itemize}
% 	\item Development policies, like using one or few \gls{version}s or many and very frequent releases;
	% \item Nature, purpose and sequence of the revision processes that are required to check the advancement status (that can be internal or external, blocking or non-blocking, ecc.);
	% \item Need to provide evidence of feasibility, with prototypes and preliminary analysis;
	% \item Evolution of the system and its requirements;
%\end{itemize}

%\subsection{Principio di miglioramento continuo}

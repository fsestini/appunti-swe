\section{Processi software}

Un \gls{processo} ingegneristico consiste in un insieme di attività correlate
atte a trasformare uno o più elementi in \frgnword{input} in un prodotto
\frgnword{output}, consumando risorse. I processi software sono ciò che
caratterizza il lavoro dell'ingegnere del software. Includono una pre ed una
postcondizione e decompongono le attività in incarichi (\frgnword{task}), la più
piccola unità di lavoro soggetta a gestione.

Così come le procedure di un programma, le attività di un processo sono
tipicamente ripetute più volte, in modo iterativo, secondo il principio del
\strong{riuso}.

% TODO Controllo di processo: trattato nella parte di qualità di processo...
% Inserire anche qua?

I processi software sono descritti dallo standard ISO 12207, dove sono
categorizzati in \strong{primari}, di \strong{supporto} e
\strong{organizzativi}.

\subsection{Processi primari}

I processi primari sono messi in atto dalle parti primare di un progetto, quali
l'acquirente, il fornitore, lo sviluppatore, l'utilizzatore e il mantenitore.

\begin{itemize}
  \item \strong{ Acquisizione }: definisce le attività svolte dal committente
    del progetto software. Una di esse è l'accettazione del prodotto finito;
  \item \strong{ Fornitura }: definisce le attività svolte dal fornitore del
    prodotto software. Tra esse figurano le attività di pianificazione e di
    consegna;
  \item \strong{ Sviluppo }: definisce le attività che costituiscono lo sviluppo
    del prodotto. Tale processo include sia le attività di analisi e
    progettazione che quelle di implementazione. Vedi
    \ref{sec:development-process}.
  \item \strong{ Gestione operativa (utilizzo) }: definisce le attività e i
    compiti dell'operatore del sistema software.
  \item \strong{ Manutenzione }: definisce le attività del mantenitore.
\end{itemize}

\subsubsection{Processo di sviluppo}\label{development-process}

Il processo di sviluppo consiste nelle seguenti attività:

\begin{itemize}
  \item \strong{Implementazione di processo}: viene selezionato un modello di
    ciclo di vita, e su esso vengono mappate le attività di sviluppo. Le
    attività del processo di sviluppo vengono pianificate;
  \item \strong{Analisi dei requisiti di sistema}: viene prodotta una SRS per il
    sistema;
  \item \strong{Progettazione del sistema}: viene stabilità una architettura
    \frgnword{top-level} del sistema, che deve soddisfare, tra il resto,
    tracciabilità e consistenza verso i requisiti di sistema;
  \item \strong{Analisi dei requisiti software}: viene prodotta una SRS che
    descrive i requisiti del prodotto software;
  \item \strong{Progettazione architetturale}: viene prodotta l'architettura
    software, a partire dai requisiti individuati;
  \item \strong{Progettazione di dettaglio}: viene prodotta una progettazione
    dettagliata e a basso livello delle singole componenti, compresa una
    definizione in dettaglio delle interazioni e interfacce tra esse;
  \item \strong{Codifica e \frgnword{ testing }}: viene sviluppato e documentato
    ogni unità software e ogni test associato ad essa. Ogni componente è testata
    per assicurare che soddisfi i requisiti;
  \item \strong{Integrazione}: viene sviluppato un piano di integrazione delle
    singole componenti in oggetti software;
  \item \strong{\frgnword{Software qualification testing}}:
    % TODO equivalente italiano?
    viene verificato che ogni componente software soddisfi i requisiti
    di qualità, secondo metriche e procedure definite dal piano di qualifica;
  \item \strong{Integrazione del sistema}: i \frgnword{configuration item} sono
    integrati nel sistema;
  \item \strong{\frgnword{System qualification testing}}:
    % TODO equivalente italiano?
    viene verificato che il sistema soddisfi i requisiti di qualità;
  \item \strong{Installazione}: viene sviluppato un piano di installazione del
    prodotto nell'ambiente designato dal contratto;
  \item \strong{Supporto all'accettazione}
\end{itemize}

Le attività possono essere eseguite in maniera iterativa e sovrapposte, e non
devono seguire uno specifico ordine temporale. I \frgnword{task} che compongono
le singole attività possono essere eseguiti in iterazioni diverse, ma devono
essere tutti portati a termine, prima o dopo. % TODO maybe tono inopportuno

\subsection{Processi di supporto}

Questi processi vengono impiegati ed eseguiti da altri quando necessario, e
costituiscono supporto al processo che li esegue. Contribuiscono al successo e
alla qualità del prodotto software.

\begin{itemize}
  \item Documentazione;
  \item Gestione di configurazione;
  \item \frgnword{Quality assurance};
  \item Verifica;
  \item Validazione;
  \item ...
\end{itemize}

\subsection{Processi organizzativi}

I processi organizzativi sono impiegati dall'azienda nell'atto di implementare e
migliorare costantemente la struttura sottostante costituita da processi e
personale.

\begin{itemize}
  \item Gestione dei processi;
  \item Gestione delle infrastrutture;
  \item Miglioramento di processo;
  \item Formazione del personale;
\end{itemize}

\subsection{Processi, aziende, progetti}

Non esistono processi migliori in generale: i processi vanno selezionati,
adattati e messi in atto appropriatamente in relazione al progetto, il contesto
aziendale e il dominio di applicazione. Un \strong{dominio di applicazione}
definisce un insieme di requisiti, terminologie e funzionalità comuni a tutti i
programmi relativi ad una particolare area dello sviluppo software. Stessi
domini applicativi condividono, ragionevolmente, gli stessi processi software.

\begin{itemize}
  \item Processo standard: riferimento di base, condiviso dalle aziende di uno
    stesso dominio applicativo.
  \item Processo definito: specializzazione di un processo standard, adattato ai
    particolari bisogni dell'azienda. Devono essere chiari, stabili, documentati
    e indipendenti dal modello di ciclo di vita, dalle tecnologie e dal dominio
    applicativo specifici.
  \item Processo di progetto: istanziazione di un processo definito. Consuma le
    risorse dell'azienda per ottenere obbiettivi prefissati e limitati nel tempo
    (progetti).
\end{itemize}

La specializzazione dei processi richiede scelte chiare e accurate, in cui è
necessario definire lo scenario di applicazione, le attività e i compiti
aggiuntivi o specifici e organizzare le relazioni tra i processi specializzati.
Tra i fattori che influiscono nella specializzazione vi sono:

\begin{itemize}
  \item Dimensione e complessità del progetto;
  \item Rischi legati al dominio applicativo e alle tecnologie in uso;
  \item Competenze ed esperianza delle risorse umane;
  \item Fattori dipendenti dal contratto in essere;
\end{itemize}

\subsection{Principio di miglioramento continuo}

L'organizzazione interna dei processi deve seguire il principio del
miglioramento continuo della qualità. Il \strong{ciclo di Deming} (o ciclo PDCA)
è un modello iterativo studiato per questo scopo, in un'ottica a lungo raggio.
Esso è strutturato in quattro stadi, quattro attività aggiuntive che realizzano
il miglioramento:

\begin{enumerate}
  \item \strong{\frgnword{Plan}}: pianificare gli obbiettivi di miglioramento;
  \item \strong{\frgnword{Do}}: eseguire tutte le attività pianificate, senza
    eccezioni. Il rigore è essenziale per essere tracciabili: le attività
    sono analizzabili solo se ripetibili;
  \item \strong{\frgnword{Check}}: verificare i risultati del processo e
    confrontarli con le aspettative;
  \item \strong{\frgnword{Act}}: agire per risolvere i problemi emersi,
    applicando correzzioni;
\end{enumerate}

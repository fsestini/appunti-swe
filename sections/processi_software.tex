\section{Processi software}
Un \gls{processo} ingegneristico consiste in un insieme di attività correlate atte a trasformare uno o più elementi in \frgnword{input} in un prodotto \frgnword{output}, consumando risorse. I processi software sono ciò che caratterizza il lavoro dell'ingegnere del software. Includono una pre- ed una postcondizione e decompongono le attività in incarichi (\frgnword{task}), la più piccola unità di lavoro soggetta a gestione. Così come le procedure di un programma, le attività di un processo sono tipicamente ripetute più volte, in modo iterativo, secondo il principio del \strong{riuso}.

\todo{Controllo di processo: trattato nella parte di qualità di processo... Inserire anche qua?}

I processi software si categorizzano in \strong{primari}, di \strong{supporto} e \strong{organizzativi}.

\paragraph{Processi primari}
I processi primari sono messi in atto dalle parti primare di un progetto, quali l'acquirente, il fornitore, lo sviluppatore, l'utilizzatore e il mantenitore. Essi sono:

\begin{itemize}
	\item Acquisizione;
	\item Fornitura;
	\item Sviluppo;
	\item Gestione operativa (utilizzo);
	\item Manutenzione;
\end{itemize}

\paragraph{Processi di supporto}
Questi processi vengono impiegati ed eseguiti da altri processi quando necessario, e costituiscono supporto al processo che li esegue. Contribuiscono al successo e alla qualità del prodotto software.

\begin{itemize}
	\item Documentazione;
	\item Gestione di configurazione;
	\item \frgnword{Quality assurance};
	\item Verifica;
	\item Validazione;
	\item Analisi congiunta (\frgnword{joint review});
	\item Revisione (\frgnword{audit});
	\item Risoluzione di problemi;
\end{itemize}

\paragraph{Processi organizzativi}
I processi organizzativi sono impiegati dall'azienda nell'atto di implementare e migliorare costantemente la struttura sottostante costituita da processi e personale.

\begin{itemize}
	\item Gestione dei processi;
	\item Gestione delle infrastrutture;
	\item Miglioramento di processo;
	\item Formazione del personale;
\end{itemize}

%\subsection{Processi, aziende, progetti}
%\subsection{Principio di miglioramento continuo}

\section{Gestione di progetto}
La gestione di progetto e' una parte essenziale dell'Ingegneria del Software. Nella maggior parte dei casi, gli obbiettivi principali di un progetto sono i seguenti:
\begin{enumerate}
	\item Consegnare il software al cliente nei tempi accordati;
	\item Mantenere i costi all'interno del \frgnword{budget};
	\item Fornire software aderente alle aspettative del committente;
	\item Mantenere un team di sviluppo organizzato ed efficiente;
\end{enumerate}

L'insegneria del software si differenzia da altri tipi di ingegneria in modi che la rendono particolarmente impegnativa:
\begin{enumerate}
	\item Il prodotto e' intangibile: i responsabili di progetto non possono rendersi conto del progresso semplicemente osservando il prodotto in via di sviluppo, ma devono affidarsi ad altre persone incaricate di produrre prove dell'avanzamento; 
	\item I progetti software di grandi dimensioni sono spesso di tipo `\frgnword{one-off}' \todo{Inserire definizione di one-off}. Pertanto, anche \frgnword{software engineers} con ampia esperienza pregressa possono avere difficolta nell'anticipare i rischi. Inoltre, il rapido avanzamento tecnologico puo' rendere questa esperienza obsoleta in breve tempo.
	\item I processi software sono variabili e specifici per ogni azienda; nonostante il progresso significativo nella standardizzazione, i processi software variano ampiamente da un'azienda all'altra;
\end{enumerate}

\todo{Manca una vera definizione di responsabile di progetto}
Il responsabile di progetto compie tipicamente le seguenti attivita':
\begin{itemize}
	\item Pianificazione di progetto: i responsabili di progetto pianificano e stimano lo sviluppo di progetto, e assegnano compiti alle persone. Essi supervisionano il lavoro e monitorano l'avanzamento;
	\item Comunicazione: i responsabili tipicamente si occupano di comunicare lo stato di avanzamento del progetto agli \frgnword{stakeholder};
	\item Gerstione dei rischi: i responsabili di progetto devono valutare i rischi che potrebbero verificarsi, monitorarli, e agire nel caso si verificassero;
	\item Gestione del personale: i responsabili hanno il compito di selezionare il team di sviluppo e stabilire metodi di lavoro efficaci;
\end{itemize}

\subsection{Gestione dei rischi}
La gestione dei rischi riguarda l'anticipazione dei rischi che potrebbero intaccare la pianificazione delle attivita' o la qualita' del prodotto, e la messa in atto di misure per evitarli.

Vi sono tre categorie di rischi, correlate tra loro:
\begin{enumerate}
	\item Rischi di progetto: influenzano la pianificazione o le risorse di progetto (ad esempio, la perdita di un bravo progettista);
	\item Rischi di prodotto: colpiscono la qualita' o le prestazioni del software (ad esempio, un componente acquistato da fornitori terzi non funziona come ci si aspetta);
	\item Rischi di impresa: influenzano l'azienda sviluppatrice o fornitrice del software (ad esempio, un la messa in vendita di un nuovo prodotto da parte di una azienda concorrente);
\end{enumerate}

La gestione dei rischi e' particolarmente importante nei progetti software data l'insita incertezza che li caratterizza, causata principalmente da requisiti poco chiari e spesso variabili nel tempo, difficolta' nello stimare il tempo e le risorse necessarie e le differenze di abilita' tra gli individui che andranno a comporre il team;

Il processo di gestione dei rischi si costituisce di alcuni passi:
\begin{enumerate}
	\item Identificazione dei rischi;
	\item Analisi dei rischi: valutare la probabilita' che possano verificarsi e le conseguenze; i risultati di questa fase vanno registrati nel piano di progetto;
	\item Pianificazione: stilare un piano per la loro gestione; 
	\item Monitoraggio dei rischi: valutare regolarmente i rischi, e come essi sono gestiti;
\end{enumerate}

Il processo di gestione dei rischi e' iterativo e continua durante tutto il progetto. I risultati del processo vanno documentati nel piano di progetto, inclusa l'analisi e le linee guida su come affrontarli, e una discussione di quelli gestiti durante il progetto.

%\subsection{Ruoli}
\todo{Scrivere da zero}
%\subsubsection{Analisti e progettisti}
%\subsubsection{Programmatori e verificatori}
%\subsubsection{Responsabile di progetto}
%\subsubsection{Amministratore}
%\subsection{Pianificazione di progetto}
%\subsubsection{Project scheduling}
%\subsubsection{Allocazione delle risorse}
%\subsubsection{Stima dei costi di progetto}
%\subsection{Piano di progetto}

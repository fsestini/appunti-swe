\section{Gestione di progetto}

La gestione di progetto e' una parte essenziale dell'Ingegneria del Software.
Nella maggior parte dei casi, gli obbiettivi principali di un progetto sono i
seguenti:

\begin{enumerate}
  \item Consegnare il software al cliente nei tempi accordati;
  \item Mantenere i costi all'interno del \frgnword{budget};
  \item Fornire software aderente alle aspettative del committente;
  \item Mantenere un team di sviluppo organizzato ed efficiente;
\end{enumerate}

L'ingegneria del software si differenzia da altri tipi di ingegneria in modi che
la rendono particolarmente impegnativa:

\begin{enumerate}
  \item Il prodotto e' intangibile: i responsabili di progetto non possono
    rendersi conto del progresso semplicemente osservando il prodotto in via di
    sviluppo, ma devono affidarsi ad altre persone incaricate di produrre prove
    dell'avanzamento;
  \item I progetti software di grandi dimensioni sono spesso di tipo
    `\frgnword{one-off}' \todo{Inserire definizione di one-off}. Pertanto, anche
    \frgnword{software engineers} con ampia esperienza pregressa possono avere
    difficolta nell'anticipare i rischi. Inoltre, il rapido avanzamento
    tecnologico puo' rendere questa esperienza obsoleta in breve tempo.
  \item I processi software sono variabili e specifici per ogni azienda;
    nonostante il progresso significativo nella standardizzazione, i processi
    software variano ampiamente da un'azienda all'altra;
\end{enumerate}

\subsection{Gestione dei rischi}

La gestione dei rischi riguarda l'anticipazione dei rischi che potrebbero
intaccare la pianificazione delle attivita' o la qualita' del prodotto, e la
messa in atto di misure per evitarli.

Vi sono tre categorie di rischi, correlate tra loro:

\begin{enumerate}
  \item \strong{Rischi di progetto}: influenzano la pianificazione o le risorse
    di progetto (ad esempio, la perdita di un bravo progettista);
  \item \strong{Rischi di prodotto}: colpiscono la qualita' o le prestazioni del
    software (ad esempio, un componente acquistato da fornitori terzi non
    funziona come ci si aspetta);
  \item \strong{Rischi di impresa}: influenzano l'azienda sviluppatrice o
    fornitrice del software (ad esempio, la messa in vendita di un nuovo
    prodotto da parte di una azienda concorrente);
\end{enumerate}

La gestione dei rischi è particolarmente importante nei progetti software data
l'insita \strong{incertezza} che li caratterizza, causata principalmente da
\strong{requisiti poco chiari} e spesso variabili nel tempo, difficolta' nello
\strong{stimare} il tempo e le risorse necessarie e le differenze di abilita'
tra gli individui che andranno a comporre il team.

Il processo di gestione dei rischi si costituisce di alcuni passi:

\begin{enumerate}
  \item \strong{Identificazione} dei rischi;
  \item \strong{Analisi} dei rischi: valutare la probabilita' che possano
    verificarsi e le conseguenze. I risultati di questa fase vanno registrati
    nel piano di progetto;
  \item \strong{Pianificazione}: stilare un piano per la loro gestione;
  \item \strong{Monitoraggio} dei rischi: valutare regolarmente i rischi, e come
    essi sono gestiti;
\end{enumerate}

Il processo di gestione dei rischi e' \strong{iterativo} e continua durante
tutto il progetto. I risultati del processo vanno documentati nel piano di
progetto, inclusa l'analisi e le linee guida su come affrontarli, e una
discussione di quelli gestiti durante il progetto.

\subsection{Ruoli}

% TODO scrivere la definizione di 'ruolo'

\subsubsection{Analisti e progettisti}

Gli \strong{analisti} conoscono il dominio del problema e possiedono una vasta
esperienza professionale; pertanto, influiscono pesantemente sul successo del
progetto. Poiche' ve ne sono pochi, raramente seguono uno stesso progetto per
tutti il ciclo di vita. \todo{frase ampiamente migliorabile}

I \strong{progettisti} hanno una conoscenza tecnica e tecnologica ampia e
aggiornata, e molta esperienza professionale. Giocano un ruolo chiave per quanto
riguarda gli aspetti tecnici e tecnologici del progetto. Sono anch'essi in
numero limitato \todo{bruttissimo}, ma talvolta seguono il progetto fino alla
manutenzione.

\subsubsection{Programmatori e verificatori}

I \strong{programmatori} partecipano alla realizzazione e alla manutenzione del
prodotto. Hanno conoscenza tecnica, visione e responsabilita' limitate. Formano
la categoria storicamente piu' popolosa.

I \strong{verificatori} partecipano all'intero ciclo di vita del software.
Possiedono competenze tecniche, esperienza di progetto, e conoscenza di leggi e
standard. Hanno capacita' di relazione e giudizio.

\subsubsection{Responsabile di progetto}

Il responsabile di progetto rappresenta l'intero progetto software nei confronti
del fornitore e del cliente. Centralizza le responsabilita' di scelta e di
approvazione, partecipa al progetto per tutta la sua durata ed e' difficile da
rimpiazzare. Ha responsabilita' nella pianificazione, nella gestione delle
risorse umane, nel controllo, nella coordinazione e nelle relazioni esterne.
Deve essere in possesso di conoscenza tecnica e abilita' nel comprendere e
anticipare l'evoluzione del progetto.

Il responsabile di progetto compie tipicamente le seguenti attivita':

\begin{itemize}
  \item \strong{Pianificazione di progetto}: i responsabili di progetto
    pianificano e stimano lo sviluppo di progetto, e assegnano compiti alle
    persone. Essi supervisionano il lavoro e monitorano l'avanzamento;
  \item \strong{Comunicazione}: i responsabili tipicamente si occupano di
    comunicare lo stato di avanzamento del progetto agli \frgnword{stakeholder};
  \item \strong{Gestione dei rischi}: i responsabili di progetto devono valutare
    i rischi che potrebbero verificarsi, monitorarli, e agire nel caso si
    verificassero;
  \item \strong{Gestione del personale}: i responsabili hanno il compito di
    selezionare il team di sviluppo e stabilire metodi di lavoro efficaci;
\end{itemize}

\subsubsection{Amministratore}

L'amministratore gestisce e controlla l'ambiente di lavoro, amministra le
risorse di progetto e le infrastrutture, gestisce la documentazione del progetto
(\frgnword{librarian}) e gli strumenti per il controllo di versione e
configurazione.

\subsection{Pianificazione di progetto}

La pianificazione di progetto è un'attività svolta dal Responsabile, e consiste
nel \strong{suddividere} il lavoro in più parti e le \strong{assegnarle} ai
membri del team di sviluppo. La pianificazione di progetto avviene in tre
momenti del ciclo di vita del progetto:

\begin{enumerate}
  \item Al momento della \strong{proposta} al committente, per aiutare a capire
    se si possiedono le risorse necessarie, e valutare il prezzo del lavoro;
  \item Durante la fase di \strong{avvio} del progetto, per pianificare la
    suddivisione del lavoro e chi vi lavorera', decidere l'allocazione delle
    risorse in azienda, ecc.;
  \item Periodicamente \strong{durante} il progetto, quando e' necessario
    effettuare modifiche al piano in luce dell'esperienza maturata e delle
    informazioni ricavate dal monitoraggio e dall'avanzamento del lavoro.
\end{enumerate}

\subsubsection{\frgnword{Project scheduling}}

% TODO Non trovo una traduzione decente di `scheduling' che sia diversa da
% `pianificazione', che gia' traduce `planning'. Rimane in inglese per ora.

\frgnword{Project scheduling} e' il processo nel quale si decide come il lavoro
verra' suddiviso in compiti separati (\frgnword{tasks}), e come e quando questi
dovranno essere eseguiti. Alcuni compiti possono essere eseguiti in parallelo,
avendo piu' persone che lavorano su componenti diversi del sistema. \`E
necessario stimare il tempo di calendario neccessario al completamento di
ciascun compito oltre allo sforzo richiesto, e assegnarlo a un componente del
team.

% TODO inserire fig. 23.4 dal Sommerville

La rappresentazione dello \frgnword{schedule} di progetto puo' servirsi di
\strong{strumenti}. Tra i piu' comuni vi sono:

\begin{itemize}
  \item Diagrammi di \strong{Gantt}: basati su calendario, mostrano chi e'
    responsabile per ogni attivita' pianificata, oltre agli istanti di inizio e
    di fine previsti;
  \item Program Evaluation and Review Technique (\strong{PERT});
  \item \strong{Work Breakdown Structure}.
\end{itemize}

Le \strong{attività di progetto} sono l'elemento di pianificazione di base. Ogni
attività è caratterizzata da:

\begin{enumerate}
  \item \strong{Durata} in giorni o mesi di calendario;
  \item \strong{Stima del lavoro} necessario al suo completamento, in termini di
    tempo/persona;
  \item \strong{\frgnword{Deadline}} entro il quale l'attività deve essere
    completata.
\end{enumerate}

\paragraph{Diagramma di Gantt}

Il diagramma di Gantt permette di organizzare le attività di progetto in
\strong{funzione del tempo}. Ogni attività è rappresentata da una barra: la
posizione e
la lunghezza della barra danno indicazione della data di inizio, di fine e della
durata dell'attività.

% TODO immagine di esempio di diagramma Gantt

Il diagramma di Gantt mostra istantaneamente \strong{cosa} deve essere fatto e
\strong{quando}. Permette di capire a vista molte informazioni, tra cui quali
attività si \strong{sovrappongono}, e la durata complessiva del progetto.
Indicando il tempo effettivo impiegato per ciascuna attività, è possibile
inoltre confrontare le stime con i progressi.

%\paragraph{PERT diagram}
%\paragraph{Work Breakdown Structure}

\subsubsection{Allocazione delle risorse}

Le attivita' vanno assegnate ai ruoli, e i ruoli alle persone. E' importante non
sovrastimare (o sottostimare) la quantita' di lavoro richiesto da ciascuna
attivita'.

\subsubsection{Stima dei costi di progetto}

Stimare la quantità di lavoro richiesto dalle attivita e calendarizzarle non è
un'operazione semplice, a causa di molteplici fattori di incertezza soprattutto
nella fase iniziale. Le aziente possono avvalersi di due tipi di tecnice durante
questa fase di progetto:

\begin{itemize}
  \item Tecniche basate sull'esperienza: la stima e' basata sull'esperienza
    maturata nei precedenti progetti dal responsabile;
  \item Modellazione algoritmica dei costi: vengono utilizzate alcune formule
    per calcolare il lavoro richiesto dal progetto in base a stime sugli
    attributi del prodotto;
\end{itemize}

Una \gls{metrica} tipicamente usata per misurare il lavoro necessario in termini
di tempo e' il \strong{tempo/persona}.

Tra i fattori che influenzano la stima vi sono:

\begin{itemize}
  \item Dimensione del progetto: quante linee di codice in relazione al
    tempo necessario a scriverle;
  \item Esperienza nel dominio applicativo;
  \item Tecnologie adottate;
  \item Ambiente di sviluppo;
  \item Livello di qualita' richiesto dai processi, in termini di efficienza ed
    efficacia;
\end{itemize}

\paragraph{Constructive Cost Model (CoCoMo)}

Il CoCoMo e' un modello empirico definito sulla base di dati raccolti nel tempo
da un largo numero di progetti software, successivamente analizzati per creare
formule che aderissero il meglio possibile alle osservazioni. CoCoMo misura le
risorse necessarie in mesi/persona (M/P).

\[
  M/P = C \times D^S \times M
\]

\begin{itemize}
  \item \strong{C}: fattore di complessita' del progetto;
  \item \strong{D}: dimensione stimata del prodotto software, espressa in KDSI
    (Kilo Delivered Source Instructions);
  \item \strong{S}: fattore di complessita';
  \item \strong{M}: moltiplicatori di costo, $\prod_i \alpha_i$ dove $\alpha_i$
    sono attributi i cui valori cadono entro interfalli fissati;
\end{itemize}

Nella sua versione di base, CoCoMo assume l'utilizzo di un modello si sviluppo
sequenziale (a cascata). \todo{da verificare}

Si possono evidenziare tre gradi di complessita':

\begin{enumerate}
  \item \strong{Semplice} (complessita' bassa): una singola persona e' in grado
    di comprendere tutto il prodotto nel suo inseme;
  \item \strong{Moderato} (media complessita'): una persona e' in grado di
    comprendere il prodotto solo isolandolo per componenti;
  \item \strong{\frgnword{embedded}} (complessita' elevata): il prodotto
    interagisce con componenti esterni e con l'ambiente;
\end{enumerate}

% TODO Inserire grafico CoCoMo

\subsection{Piano di progetto}

Il piano di progetto e' un documento che espone le risorse disponibili nel
progetto, il \frgnword{work breakdown} e il calendario delle attivita', oltre
che il risultato del processo di analisi dei rischi. Viene costantemennte
aggiornato durante tutto il ciclo di vita, e ha lo scopo di organizzare le
attivita' in modo da permettere l'efficace valutazione del progresso raggiunto
dal progetto in ogni dato momento.

Nel piano di progetto vengono stabilite le \frgnword{milestone}, punti critici
all'interno del calendario delle attività su cui è possibile valutare il
progresso raggiunto.

Un piano di progetto ha la seguente struttura tipica:

\begin{itemize}
  \item Introduzione (obbiettivi e struttura del progetto);
  \item Organizzazione del progetto;
  \item Analisi dei rischi;
  \item Risorse disponibili e necessarie;
  \item Scomposizione delle attivita' (\frgnword{work breakdown});
  \item Calendario delle attivita';
  \item Controllo e rendicontazione;
\end{itemize}

\section{Progettazione software}

Per perseguire la correttezza per costruzione, si progetta \strong{prima}
di produrre.

È necessario progettare per

\begin{itemize}
  \item governare la complessità del prodotto
  \item organizzare e ripartire le responsabilità di realizzazione
  \item produrre in economia (efficienza) e garantire controllo di qualità
        (efficacia)
\end{itemize}

Mentre l'analisi segue un processo \strong{investigativo}, la progettazione
segue un approccio \strong{sintetico} nel quale da vincoli, obbiettivi e
requisiti si ricava una soluzione soddisfacente per tutti gli
\foreignword{stakeholder}.

%TODO: da verificare
%La progettazione architetturale è il primo \foreignord{stage} nel processo di
%progettazione software.

\subsection{Architettura software}

L'attività di progettazione fissa l'\strong{architettura} software del prodotto,
definita come

\begin{itemize}
  \item decomposizione del sistema in componenti; prima visione logica
        (concettuale), poi di dettaglio (realizzativa)
  \item organizzazione dei diversi componenti, che ne definisce ruoli,
        responsabilità e interazioni
  \item interfaccie necessarie all'interazione tra le componenti, tra loro e con
        l'ambiente
  \item paradigmi di composizione delle componenti, come definizione di regole,
        criteri, limiti e vincoli
\end{itemize}

% TODO espandere
L'architettura software è importante perchè influisce sulle caratteristiche non
funzionali del sistema (mantenibilità, \foreignword{performance}, ecc.)

Una buona architettura facilita il successo del prodotto software:

\begin{itemize}
  \item impiega componenti con specifica chiara e coesa
  \item realizzabili con risorse date e costi fissati
  \item ha una struttura che facilita eventuali cambiamenti futuri
\end{itemize}

\section{Verifica e validazione}
\label{sec:verifica_e_validazione}

\begin{itemize}
  \item \strong{Verifica}: \frgnword{Did I build the right system?}
  \item \strong{Validazione}: \frgnword{Did I build the system right?}
\end{itemize}

La risposta a entrambe le domande richiede una prova oggettiva, non deve
provenire da percezioni soggettive; una delle caratteristiche dell'ingegneria
del software è l'essere \strong{quantificabile}.

\paragraph{Verifica}
\label{par:verifica}

Fornisce prove oggettive che i prodotti uscenti da una particolare fase del
ciclo di vita del software raggiungono tutti i requisiti specificati per quella
fase, portando ad un avanzamento di \frgnword{baseline}.

La verifica valuta la coerenza, completezza e correttezza del software e
relativa documentazione, durante il suo sviluppo, e fornisce \strong{supporto}
per la successiva attività di validazione.

La verifica pone l'attenzione sul modo di lavorare, sul \strong{metodo}: ha a
che fare con regole, norme, procedure e strumenti. Lo svolgimento di attività
secondo regole elimina la possibilità di introdurre errori, a patto che
tali regole siano complete e prive di buchi.

Il verificatore \strong{non deve} svolgere lavoro già eseguito da altri. Devono
esserci norme e procedure che assicurino che il prodotto è corretto, e norme che
assicurino che è stato verificato.

\paragraph{Validazione}
\label{par:validazione}

Conferma, data da ispezione e prove oggettive, che il software è conforme ai
bisogni dell'utente, e che i requisiti sono completamente soddisfatti.

La validazione si può attuare, per definizione, solo a fine progetto, quando è
disponibile un sistema da validare (coincide spesso con l'attività di test di
sistema.) Si avranno tante validazioni quanti sono i prodotti realizzati.

La validazione è supportata dalla verifica. La validazione dice che la risposta
è sensata, ma l'assicurare che la validazione sia in grado di mappare la
risposta e la domanda completamente è compito della verifica.

\subsection{Analisi statica}
\label{sub:analisi_statica}

L'analisi statica non richiede l'esecuzione di codice. Studia caratteristiche
del codice sorgente (talvolta anche del codice oggetto) e della documentazione
associata, valutando la conformità alle regole.

Viene utilizzata prevalentemente in \strong{verifica}, quando non è disponibile
un prodotto finito.

Un numero crescente di sistemi software incorpora funzionalità critiche in
termini di \strong{sicurezza}, intesa come

\begin{itemize}
  \item \strong{\frgnword{Safety}}: prevenzione di condizioni di pericolo verso
  persone o cose;
  \item \strong{\frgnword{Security}}: prevenzione di intrusioni.
\end{itemize}

Tali sistemi devono possedere tutte le caratteristiche funzionali e non
funzionali previste dai requisiti, e servono prove dimostrabili del possesso di
tali proprietà.

Nessun linguaggio garantisce a priori la completà verificabilità dei programmi
scritti in esso. In ogni linguaggio c'è compromesso tra potere espressivo e
costo di verifica. Tale compromesso va considerato nella scelta del linguaggio
e dei suoi costrutti.

\subsubsection{Forme di analisi statica}
\label{ssub:forme_di_analisi_statica}

L'analisi statica non richiede esecuzione di parti del sistema, e si applica a
qualunque prodotto di processo, non solo il codice.

\begin{itemize}
  \item \strong{Metodi di lettura} (\frgnword{desk check}), impiegati solo per
    prodotti semplici;
  \item \strong{Metodi formali}, basati sulla prova assistita di proprietà la cui
    dimostrazione dinamica può essere eccessivamente onerosa.
\end{itemize}

\paragraph{Metodi di lettura}
\label{par:metodi_di_lettura}

I metodi di lettura solo \strong{\frgnword{walkthrough}} e
\strong{\frgnword{inspection}}. Sono pratici, basati sulla lettura del documento
e dipendenti dall'esperienza del verificatore.

L'\frgnword{inspection} consiste nel rilevare la presenta di difetti eseguendo
una lettura mirata della documentazione (codice e non), focalizzando la ricerca
su presupposti (\frgnword{error guessing}.) Viene svolta da verificatori
separati e distinti dai programmatori, e si articola nei seguenti passi:

\begin{itemize}
  \item Pianificazione;
  \item Definizione della lista di controllo;
  \item Lettura;
  \item Correzione dei difetti;
  \item Redazione di documentazione come rapporto delle attività svolte.
\end{itemize}


Il \frgnword{walkthrough} consiste nel rilevare la presenza di difetti eseguendo
una lettura a largo spettro della documentazione, senza assunzione di
presupposti. Viene svolta da agenti e sviluppatori, ma con ruoli ben distinti.
Si articola nei seguenti passi:

\begin{itemize}
  \item Pianificazione;
  \item Lettura;
  \item Discussione;
  \item Correzione degli errori;
  \item Redazione di documentazione come rapporto delle attività svolte.
\end{itemize}

Una possibile strategia per il \frgnword{walkthrough} del codice è la sua
percorrenza simulandone l'esecuzione.

Il \frgnword{walkthrough} è il metodo più semplice, ma è oneroso e non troppo
utile. Richiede maggiore attenzione e tempo, mentre l'\frgnword{inspection}
risulta la tecnica più rapida e, laddove non sia possibile automatizzare la
verifica, appare la più ragionevole.

% TODO subsection 'Verifica e validazione di qualità'

\subsubsection{Tecniche di verifica}
\label{ssub:tecniche_di_verifica}

\paragraph{Tracciamento}
\label{par:tracciamento}

% paragraph tracciamento (end)

Il tracciamento dimostra completezza ed economicità della soluzione (la
soluzione ha \strong{tutto} e \strong{solo} quello che si doveva fare. Nessun
componente è inguistificato.)

Il tracciamento ha luogo

\begin{itemize}
  \item tra requisiti software e requisiti utente;
  \item tra procedure di verifica e requisiti, disegno, codice;
  \item tra codice sorgente e codice oggetto.
\end{itemize}

Alcuni stili di verifica facilitano il tracciamento. L'obiettivo è assegnare
singoli requisiti elementari a singoli moduli del programma così da
richiedere una sola procedura di prova e ottenere una più semplice
corrispondenza tra essi. Maggiore è l'astrazione di un costrutto del linguaggio
maggiore la quantità di codice oggetto generato e maggiore l'onere di
dimostrazione di corrispondenza.

\paragraph{Revisioni (\frgnword{joint review / audit})}
\label{par:joint_review_audit}

Le revisioni sono uno strumento essenziale del processo di verifica. Possono
essere condotte su analisi, progettazione, codice, procedure e risultati di
verifica. Richiedono l'interazione tra individui, pertanto non sono
automatizzabili.

Le revisioni possono essere \strong{formali} (\frgnword{audit}) o
\strong{informali} (\frgnword{joint review}), richiedono indipendenza tra
verificato e verificatore e semplicità e leggibilità di codice e diagrammi.

\subsubsection{Tipi di analisi statica}
\label{ssub:tipi_di_analisi_statica}

\paragraph{Analisi di flusso di controllo}
\label{par:analisi_di_flusso_di_controllo}

L'analisi di flusso di controllo permette di

\begin{itemize}
  \item Accertare che il codice esegua nella sequenza specificata e sia ben
  strutturato;
  \item Localizzare codice non raggiungibile o loop infiniti.
\end{itemize}

\paragraph{Analisi di flusso dei dati}
\label{par:analisi_di_flusso_dei_dati}

L'analisi di flusso dei dati permette di

\begin{itemize}
  \item Accertare che nessun cammino di esecuzione acceda a variabili prive di
  valore;
  \item Rilevare possibili anomalie (es. più scritture successive senza letture
  intermedie);
  \item Problemi legati a variabili globali.
\end{itemize}

\paragraph{Analisi di flusso di informazione}
\label{par:analisi_di_flusso_di_informazione}

Permette di determinare quali dipendenze tra ingressi e uscite risultino
dall'esecuzione di una unità di codice. Le sole dipendenze consentite sono
quelle previste dalla specifica.

L'analisi può limitarsi a un singolo modulo o estendersi a più moduli correlati
o all'intero sistema.

\paragraph{Esecuzione simbolica}
\label{par:esecuzione_simbolica}

Metodo per analizzare quali input causano l'esecuzione delle diverse parti di un
programma. Un interprete esegue il programma, associando simboli a valori in
input (invece che usare valori provenienti dall'effettiva esecuzione) e
arrivando quindi a descrivere il programma come espressioni e vincoli in termini
di tali simboli.

Permette di verificare proprietà del programma mediante manipolazione algebrica
del codice sorgente, combinando tecniche di analisi di flusso di controllo, di
flusso di dati e di flusso di informazione.

\paragraph{Verifica formale del codice}
\label{par:verifica_formale_del_codice}

La verifica formale è atta a provare la correttezza del codice sorgente
rispetto alla specifica algebrica dei
requisiti. Le condizioni di verifica sono espresse come teoremi la cui verità
implica certe pre-condizioni in ingresso e certe post-condizioni in uscita.

Sono ipotesi di terminazione del programma, si ha una prova di correttezza
parziale. Una prova totale richiede prova di terminazione.

\paragraph{Analisi di limite}
\label{par:analisi_di_limite}

Verifica che i dati del programma rimangano entro i limiti di tipo e di
precisione desiderati. Vengono verificati eventuali

\begin{itemize}
  \item Overflow e underflow;
  \item Errori di arrotondamento;
  \item Range checking;
  \item Analisi di limite di strutture.
\end{itemize}

Linguaggi evoluti assegnano limiti statici a tipi discreti consentendo verifiche
automatiche sulle corrispondenti variabili. La verifica è più problematica con
tipi enumerati e reali.

\paragraph{Analisi di uso dello \frgnword{stack}}
\label{par:analisi_di_uso_dello_frgnword_stack}

L'analisi dello stack determina la massima domanda di stack richiesta da
un'esecuzione in relazione con la dimensione dell'area di memoria assegnata al
processo, verificando che non sia possibile la collisione tra stack e heap.

\paragraph{Analisi temporale}
\label{par:analisi_temporale}

Studiare le proprietà temporali richieste ed esibite dalle dipendenze delle
uscite dagli ingressi del programma. Limiti espressivi dei linguaggi limitano
questa analisi (iterazioni while prive di limite statico, ricorso a strutture
dinamiche).

\paragraph{Analisi d'interferenza}
\label{par:analisi_d_interferenza}

L'analisi mostra l'assenza di effetti di interferenza tra parti separate del
sistema (``partizioni''.) Veicoli di interferenza tipici sono aree di memoria
dinamica (es. \frgnword{memory leak}) e dispositivi di I/O programmabili con
effetto sull'intero sistema (es. \frgnword{watchdog}.)

\paragraph{Analisi di codice oggetto}
\label{par:analisi_di_codice_oggetto}

Assicura che il codice oggetto da eseguire sia una traduzione corretta del
codice sorgente corrispondente e che nessun errore od omissione siano stati
introdotti dal compilatore.

\subsubsection{Programmi verificabili}
\label{ssub:programmi_verificabili}

Serve coerenza tra standard di codifica e costrutti del linguaggio scelti e i
metodi di verifica adottati. \`E inoltre importante sviluppare tenendo a mente
le esigenze di verifica. La verifica retrospettiva (a valle dello sviluppo) è
spesso inadeguata, e il costo di rilevazione e correzione di un errore è tanto
maggiore quanto più avanzato è lo stadio di sviluppo.

Eseguire cicli di revisione-verifica dopo ogni rilevazione di errore è
inefficace e oneroso. Meglio effettuare analisi statica durante la codifica
(\frgnword{correctness by construction}.)

\`E bene avere norme di progetto che prediligano o proibiscano l'uso di
particolari costrutti del linguaggio di programmazione, per

\begin{itemize}
  \item Assicurare comportamento predicibile;
  \item Consentire analisi del sistema;
  \item Facilitare le prove.
\end{itemize}

\paragraph{Comportamento predicibile}
\label{ssub:comportamento_predicibile}

Il codice sorgente non deve presentare ambiguità. In particolare, occore
prestare attenzione a

\begin{itemize}
  \item \frgnword{Side effects} di funzioni;
  \item Ordine di elaborazione e inizializzazione (es. ordine di attivazione di
    thread);
  \item Modalità di passaggio di parametri (per valore, per riferimento).
\end{itemize}

\paragraph{Analizzabilità del sistema}
\label{ssub:analizzabilit_del_sistema}

L'analisi statica costruisce modelli astratti del software in esame come grafo
diretto e ne studia i possibili cammini. La presenza di flussi di eccezione e di
risoluzione dinamica di chiamata (dispatching) complica notevolmente la
struttura del grafo.

Ciascun flusso di controllo (thread) viene analizzato separatamente.

\paragraph{Facilità di prova}
\label{ssub:facilit_di_prova}

La strategia di prova può essere

\begin{itemize}
  \item Investigativa, informale e senza obiettivi prefissati di copertura;
  \item Formale, regolata da norme e grado di copertura fissato.
\end{itemize}

Alcuni costrutti complicano le prove.

\begin{itemize}
  \item Il \frgnword{dispatching} complica le prove di copertura;
  \item Il \frgnword{casting} complica l'analisi dell'identità dei dati;
  \item Le eccezioni predefinite complicano le prove di copertura.
\end{itemize}

\subsubsection{Criteri di programmazione}
\label{ssub:criteri_di_programmazione}

\begin{itemize}
  \item La programmazione deve riflettere l'architettura del codice, esprimendo
  componenti, moduli, unità, e facilitarne il riuso;
  \item Separare le interfacce dall'implementazione;
  \item Massimizzare l'incapsulazione e l'\frgnword{information hiding};
  \item Usare tipi specializzati per specificare dati;
\end{itemize}

\subsubsection{Considerazioni pragmatiche}
\label{ssub:considerazioni_pragmatiche}

L'efficacia dei metodi di analisi è funzione della qualità di strutturazione del
codice.

La verifica di un programma relaziona frammenti di codice con frammenti di
specifica. La verificabilità è funzione inversa della dimensione del contesto
(conviene confinare gli ambiti e la visibilità). Una buona architettura facilita
la verifica (incapsulazione dello stato e controllo dell'accesso).

% TODO migliorare


\subsection{Analisi dinamica}
\label{sub:analisi_dinamica}

L'analisi dinamica è un metodo sia di verifica che di validazione che richiede
l'esecuzione del programma analizzato. Si svolge tramite \strong{test}, una
prova consistente nella verifica dinamica del comportamento del programma.

``Il processo di eseguire un programma con l'intento di trovarvi difetti'':
visione delle prove che rappresenta lo spirito critico e l'atteggiamento
scettico per metodo alla base di una strategia efficace di prova.

La verifica avviene su un insieme finito di \strong{casi di prova} selezionati
nel dominio delle esecuzioni possibili, che è generalmente infinito. Ciascun
caso di prova specifica i valori di ingresso e lo stato iniziale del sistema, e
deve produrre un esito decidibile (problema dell'oracolo.) Ogni caso di prova è
verificato rispetto a un \strong{comportamento atteso}.

La \strong{ripetibilità} è requisito essenziale per l'analisi dinamica, e
riguarda:

\begin{itemize}
  \item Ambiente (HW, stato iniziale)
  \item Specifica (valori in ingresso, comportamenti attesi)
  \item Procedure (esecuzione, analisi dei risultati)
\end{itemize}

Gli strumenti a supporto dell'analisi dinamica sono

\begin{itemize}
  \item \strong{Driver}: componente attiva fittizia per pilotare una parte
  \item \strong{Stub}: componente passiva fittizia per simulare una parte
  \item \strong{Logger}: componente non intrusivo di registrazione dei dati di
    esecuzione per analisi dei risultati
\end{itemize}

\subsubsection{Definizioni}
\label{ssub:definizioni}

\begin{description}
  \item[Unità] La più piccola quantità di software che è conveniente verificare
  singolarmente, tipicamente assegnata ad un singolo programmatore. La sua
  natura specifica dipende dal linguaggio di programmazione in uso. Va sempre
  intesa in senso architetturale: non linee di codice, ma entità di
  strutturazione (procedura, classe, package);
  \item[Modulo] Parte dell'unità;
  \item[Componente] Integra più unità.
\end{description}

% TODO subsection 'stub e driver'

\subsubsection{\frgnword{Faults}, \frgnword{errors} e \frgnword{failures}}
\label{ssub:terminologia}

Un \frgnword{failure} avviene quando il comportamento del sistema devia da
quanto specificato per esso. Sono il risultato di problemi interni al sistema
che ad un certo punto si manifestano nel suo comportamento esterno. Tali
problemi sono chiamati \strong{errori}, e le loro cause meccaniche, algoritmiche
o concettuali sono chiamate \strong{fault}.

Gli \emph{errori} sono stati del sistema; i \emph{fault} sono ciò che causa
l'esistenza di un errore.

Un test rileva malfunzionamenti (failures) indicando la presenza di guasti
(faults), ma non può provarne l'assenza!

% TODO elaborare

\subsubsection{Fattori da bilanciare}
\label{ssub:fattori_da_bilanciare}

Nella definizione della strategia di prova occorre un bilanciamento tra

\begin{itemize}
  \item la \strong{quantità minima di casi} di prova sufficienti a
    fornire certezza adeguate sulla qualità del prodotto (fattore
    governato da criteri tecnici);
  \item la \strong{quantità massima di sforzo}, tempo e risorse
    disponibile per il completamento della verifica (fattore governato
    da criteri gestionali).
\end{itemize}

Secondo la legge del rendimento decrescente (\frgnword{diminishing returns}).

\subsubsection{Criteri guida}
\label{ssub:criteri_guida}

L'\strong{oggetto} della prova può essere

\begin{itemize}
  \item Il sistema nel suo complesso
  \item Parti di esso, in relazione funzionale, d'uso, di comportamento, di
  struttura;
  \item Singole unità.
\end{itemize}

L'\strong{obiettivo} della prova può essere

\begin{itemize}
  \item Specificato per ogni caso di prova
  \item In termini precisi e quantitativi
  \item Variabile al variare dell'oggetto della prova;
\end{itemize}

Il Piano di Qualità risponde alla domanda ``quali e quante prove''.

La ``provabilità'' del software va assicurata a monte: la progettazione va
raffinata per assicurare provabilità.

I test devono essere ripetibili.

Una singola prova non basta

I test hanno limiti e problemi

\begin{itemize}
  \item Non esiste un algoritmo che, dato un programma P qualsiasi, eneri per
    esso un test finito ideale (definito da un criterio affidabile e valido)
    (teorema di Howden);
  \item Il test di un programma può rilevare la presenza di malfunzionamenti, ma
    non dimostrarne l'assenza (tesi di Dijkstra).
\end{itemize}

\subsubsection{Principi del testing software (secondo Bertrand Meyer)}
\label{ssub:principi_del_testing_software}

\begin{itemize}
  \item Testare un programma è cercare di farlo fallire;
  \item I test non sostituiscono le specifiche
  \item Da ogni esecuzione fallita deve essere ricavato un caso di prova da
    includere permanentemente nel progetto;
  \item Gli oracoli dovrebbero essere parte del codice, come contratti
    (???wat???)
  \item Ogni strategia di testing dovrebbe includere un processo di testing
    riproducibile ed essere valutato oggettivamente secondo criteri espliciti;
  \item La qualità più importante di una strategia di testing è il numero di
    guasti rilevati in funzione del tempo.
\end{itemize}

\subsubsection{Elementi di una prova}
\label{ssub:elementi_di_una_prova}

\begin{description}
  \item[Caso di prova (test case)] Tripla ${ingresso, uscita,
    ambiente}$. L'ambiente includo l'oggetto della prova;
  \item[Batteria di prove (test suite)] Insieme di casi di prova;
  \item[Procedura di prova] Procedimento (automatico o manuale) per eseguire,
    registrare, analizzare e valutare i risultati di prove;
  \item[Prova] $={Procedura, caso di prova}$.
  \item[Oracolo] Metodo per generare a priori i risultati attesi e convalidare i
    risultati ottenuti. Generalmente applicato  da agenti automatici per
    velocizzare la convalida e renderla ``oggettiva''.
\end{description}

\paragraph{Oracolo}
\label{par:oracolo}

In software testing, un oracolo è un meccanismo per determinare se un test è
superato o meno. Comporta il confronto tra l'output del sistema testato, dato
uno specofico input per il caso di prova, e l'output che l'oracolo determina
essere quello corretto.

Il problema dell'oracolo è spesso più complesso di quanto si possa pensare.
Alcuni oracoli sono basati su

\begin{itemize}
  \item Postcondizioni delle funzioni;
  \item Specifiche e documentazione;
  \item Altri prodotti utilizzanti algoritmi diversi da quello sotto test;
  \item Euristiche;
  \item valutazione umana.
\end{itemize}

\subsubsection{Strategie di integrazione}
\label{ssub:strategie_di_integrazione}

\begin{itemize}
  \item Assemblare parti in \strong{modo incrementale}: Aggiungo ad una
    cosa che so essere corretta un'altra cosa corretta. I difetti rilevati in un
    test sono più probabilmente da attribuirsi all'ultima parte aggiunta;
  \item Assemblare \strong{produttori prima dei consumatori}: la verifica dei
    primi garantisce ai secondi flusso di controllo (chiamate) e flusso dei dati
    corretti;
  \item Assemblare in modo che ogni passo di integrazione sia
    \strong{reversibile}, consentendo di retrocedere verso uno stato noto e
    sicuro.
\end{itemize}

La pianificazione della codifica deve essere tale da rendere disponibile le
componenti da integrare nell'ordine appropriato.

\paragraph{Bottom-up}
\label{par:bottom_up}

Si sviluppano e si integrano prima le parti con minore dipendenza funzionale e
maggiore utilità, e si procede risalendo l'albero delle dipendenze. Questo
riduce il numero di stub necessari ai test, ma non permette di avere
funzionalità tangibili nell'immediato (quanto ciò importi dipende dal rapporto
con il committente.)

\paragraph{Top-down}
\label{par:top_down}

Si sviluppano prima le parti più esterne, quelle poste sulle foglie dell'albero
delle dipendenze e poi si scende. Questa strategia comporta l'uso di molti stub
ma integra a partire dalla funzionalità più di alto livello.

\subsubsection{Classi di equivalenza}
\label{ssub:classi_di_equivalenza}

Dato il costo elevato dei test, è importante cercare di ridurre al minimo la
quantità di casi di prova necessari. Questo è possibile individuando delle
\strong{classi di equivalenza} per i valori di input.

Una classe di equivalenza è un insieme rappresentabile da un singolo valore che
ne descriva le caratteristiche. L'idea è di raggruppare in astratto valori che
ai fini di test sono indistinguibili. Ragionando per classi di equivalenza non è
necessario testare con input che coprono tutto lo spazio dei valori, ma
solamente un campione per classe di equivalenza.

Il dominio di valori per ogni parametro è determinato dal tipo del parametro e
dalla logica della funzione. Su questo dominio si individiano diverse classi di
equivalenza, che tipicamente sono:

\begin{itemize}
  \item Valori nominali interni al dominio
  \item Valori legali di limite
  \item Valori illegali
\end{itemize}

Si consideri l'esempio di una variabile che rappresenta l'indice di un array. I
valori illegali sono rappresentati dagli indici inferiori al lower bound e
superiori all'upper bound, con quelli legali all'interno. Si individuano quindi
cinque classi di equivalenza.

\subsubsection{Test di unità}
\label{ssub:test_di_unita}

L'oggetto del test di unità è l'\strong{unità}, elemento software composto da
uno o più moduli. Un modulo è un componenti elementari di progetto di dettaglio,
che non necessita quindi di test.

Unità e moduli sono determinati in fase di progettazione di dettaglio, e di
conseguenza il piano di test di unità. Il testing di unità termina quando tutte
le unità sono verificate.

Due terzi dei difetti identificati tramite analisi dinamica provengono da test
di unità, e il 50 \% di essi sono identificati da prove strutturali
(\frgnword{white box}).

Tra le diverse tipologie di test, delli di unità hanno costo maggiore. La loro
produzione è responsabilità del programmatore per unità molto semplici,
altrimenti di un verificatore indipendente.

\paragraph{Test funzionale (\frgnword{black box})}
\label{par:test_funzionale}

Il test funzionale fa riferimento alla specifica dell'unità e valuta i risultati
in uscita dall'unità, senza valutare la logica interna.

Ciascun insieme di dati di ingresso che producono un dato comportamento
funzionale costituisce un caso di prova. La scelta di classi di equivalenza
minimizza i casi di prova da considerare (valori nella medesima classe producono
lo stesso comportamento).

Il test funzionale da solo non può accertare correttezza e completezza della
logica interna dell'unità, e va necessariamente integrato con test strutturale.
Inoltre, una valutazione più efficace delle classi di equivalenza degli input
tiene conto della logica interna.

\paragraph{Test strutturale (\frgnword{white box})}
\label{par:test_strutturale}

Il test strutturale verifica la logica interna del codice dell'unità cercando
massima copertura. Ciascuna prova deve essere progettata per attivare ogni
cammino di esecuzione all'interno del modulo.

Ciascun insieme di dati di ingresso che attivano un percorso costituiscono un
caso di prova. L'uso di debugger può agevolarne l'esecuzione ma non esonera
dalla progettazione dei casi di prova.

\subsubsection{Test di integrazione}
\label{ssub:test_di_integrazione}

Il test di integrazione attua la costruzione e verifica incrementale del sistema
completo. Il suo obbiettivo è verificare che componenti verificate essere
localmente corrette, lo sono assemblate insieme.

I test di integrazione rilevano i seguenti problemi:

\begin{itemize}
  \item Errori residui nella realizzazione dei componenti;
  \item Modifica delle interfacce o cambiamenti nei requisiti;
  \item Riuso di componenti dal comportamento oscuso o inadatto;
  \item Integrazione con altre applicazioni non ben conosciute.
\end{itemize}

Una logica di integrazione funzionale:

\begin{itemize}
  \item Seleziona le funzionalità da integrare;
  \item Identifica le componenti che svolgono quelle funzionalità;
  \item Ordina le componenti per numero di dipendenze crescente;
  \item Esegue l'integrazione in quell'ordine.
\end{itemize}

Un'architettura con componenti coese e disaccoppiate facilita l'integrazione.

I test di integrazione possono rilevare difetti di progettazione o insufficiente
qualità dei test di unità.

Quanti test di integrazione servono?

\begin{itemize}
  \item Accertare che tutti i dati scambiati attraverso \strong{ciascun
    interfaccia} siano conformi alla loro specifica;
  \item Accertare che tutti i flussi di controllo previsti in specifica siano
    stati effettivamente realizzati e provati.
\end{itemize}

\subsubsection{Test di sistema}
\label{ssub:test_di_sistema}

Il test di sistema verifica il comportamento dinamico del sistema completo
rispetto ai requisiti software. Ha inizio con il completamento del test di
integrazione. Un test di sistema è completo quando ho verificato tutti i
requisiti.

\`E inerentemente funzionale (\frgnword{black box}): non dovrebbe richiedere
conoscenza della logica interna del software.

\subsubsection{Test di regressione}
\label{ssub:test_di_regressione}

Consiste nella ripetizione selettiva di test di sistema o test di integrazione,
con l'obiettivo di accertare che modifiche intervenute per correzione o
estensione di parti non comportino errori in altre parti del sistema.
Modifiche effettuate per aggiunta, correzione o rimozione non devono
pregiudicare le funzionalità già verificate. Il rischio di questo fenomeno
aumenta all'aumentare dell'accoppiamento e al diminuire dell'incapsulazione.

Il test di regressione può essere molto oneroso. La sua necessità, però, si
elimina solo con il completo isolamento, che nella pratica è irrealizzabile.

Dopo una modifica, è necessario valutare il perimetro dei test di regressione,
ovvero le componenti che potrbbero essere influenzate dalla modifica e su cui
vanno ripetuti i test. Tale perimetro è valutato architetturalmente, in base
alle dipendenze.

\subsubsection{Test di accettazione (collaudo)}
\label{ssub:test_di_accettazione}

Al test di sistema segue il \strong{collaudo}, attività supervisionata dal
committente per dimostrare la conformità del prodotto sulla base di casi di
prova specificati nel o implicati dal contratto.

Il collaudo ha \strong{implicazioni contrattuali}. \`E un'attività formale a cui
segue il rilascio del prodotto e la fine della commessa.

\subsubsection{Fattore di copertura}
\label{ssub:fattore_di_copertura}

Un test non viene valutato solamente in base al numero di fault rilevati, ma
anche con un indicatore chiamato \emph{copertura}. Il fattore di copertura
rappresenta quanto la prova esercita il prodotto, e si ricerca soprattutto in
sede di test di unità, in modo da integrare entità che già presentano coperturà
elevata.

\begin{itemize}
  \item \strong{copertura funzionale}: rispetto alla percentuale di funzionalità
    esercitate come viste dall'esterno.
  \item \strong{copertura strutturale} (branch, condition): rispetto alla
    percentuale di logica interna del codice esercitata.
\end{itemize}

Si ha \strong{statement coverage} al 100\% quando i test effettuati sull'unità
sono sufficienti a eseguire---complessivamente---almeno una volta tutte le linee
di comando di ciascuno dei moduli dell'unità.

Si ha \strong{branch coverage} al 100\% quando ciascun ramo del flusso di
controllo viene attraversato---complessivamente---almeno una volta.

Lo statement coverage è meno potente del branch coverage: non vengono rilevati
fault derivanti da statement posti in rami del flusso di controllo errati.

La copertura costituisce una misura della bontà della prova, ma un valore del
100\% non dimostra l'assenza di difetti, e può essere irraggiungibile.

\paragraph{Condizione e decisione}
\label{par:condizione_e_decisione}

\begin{itemize}
  \item \strong{Condizione}: espressione booleana semplice non contenente al suo
    interno ulteriori condizioni combinate da operatori booleani;
  \item \strong{Decisione}: espressione composta contenente condizioni combinate
    da operatori booleani.
\end{itemize}

Al crescere del numero di condizioni all'interno di una decisione il numero di
test necessario a massimizzzare il condition coverage diventa proibitivo. \`E
importante quindi massimizzare il \frgnword{modified decision condition
coverage}.

% approfondire modified decision condition coverage

\subsubsection{Maturità del prodotto}
\label{ssub:maturit_del_prodotto}

Valutare il grado di evoluzione del prodotto
\begin{itemize}
  \item quanto il prodotto migliora in seguito alle prove
  \item quanto diminuisce la densità dei difetti
  \item quanto può costare la scoperta del prossimo difetto
\end{itemize}

le tecniche correnti sono spesso empiriche, sotto influenza del modello
code-and-fix.

definire un modello ideale
modello base: il numero di difetti del software è una costante iniziale.
modello logaritmico: le modifiche introducono difetti.



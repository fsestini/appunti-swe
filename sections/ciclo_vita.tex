\section{Ciclo di vita del software}
Il ciclo di vita di un prodotto software può essere visto come una macchina a stati, dove ogni stato rappresenta un preciso grado di maturazione del prodotto, e dove ogni transizione tra stati richiede l'esecuzione di attività, raggruppate in processi. Ogni prodotto software, lungo il suo ciclo di vita, attraversa gli stati di concepimento, sviluppo, utilizzo e ritiro.

Stati e transizioni hanno precise pre- e postcondizioni, determinate da vincoli, regole e strategie dedicate. La durata temporale entro uno stato di ciclo di vita o in una transizione tra essi viene detta \strong{fase}. Le fasi mostrano l'avanzamento del progetto in funzione del tempo trascorso.

\subsection{Modelli di ciclo di vita}
I modelli di ciclo di vita del software enfatizzano i processi chiave da attuare e le relazioni e interdipendenze logiche e temporali tra di essi. Il modello di ciclo di vita adottato pone vincoli su pianificazione e gestione del progetto. Esso è indipendente da metodi e strumenti di sviluppo, e precede la loro selezione. L'ingegnere del software deve essere a conoscenza del ciclo di vita da adottare in modo da stimare costi, tempi, vincoli e benefici fin dall'inizio.

%\subsubsection{Modello sequenziale (%\frgnword{Waterfall model})}
%\subsubsection{Modello iterativo}
%\subsubsection{Modello incrementale}
%\subsubsection{Modello evolutivo}
%\subsubsection{Modello a spirale}
%\subsubsection{Modello a componenti}
%\subsubsection{Modelli agili}
%\subsection{Standard di processo}
%\subsubsection{ISO/IEC 12207:1995}

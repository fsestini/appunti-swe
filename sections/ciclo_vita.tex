\section{Ciclo di vita del software}
Il ciclo di vita di un prodotto software può essere visto come una macchina a stati, dove ogni stato rappresenta un preciso grado di maturazione del prodotto, e dove ogni transizione tra stati richiede l'esecuzione di attività, raggruppate in processi. Ogni prodotto software, lungo il suo ciclo di vita, attraversa gli stati di concepimento, sviluppo, utilizzo e ritiro.

Stati e transizioni hanno precise pre- e postcondizioni, determinate da vincoli, regole e strategie dedicate. La durata temporale entro uno stato di ciclo di vita o in una transizione tra essi viene detta \strong{fase}. Le fasi mostrano l'avanzamento del progetto in funzione del tempo trascorso.

\subsection{Modelli di ciclo di vita}
I modelli di ciclo di vita del software enfatizzano i processi chiave da attuare e le relazioni e interdipendenze logiche e temporali tra di essi. Il modello di ciclo di vita adottato pone vincoli su pianificazione e gestione del progetto. Esso è indipendente da metodi e strumenti di sviluppo, e precede la loro selezione. L'ingegnere del software deve essere a conoscenza del ciclo di vita da adottare in modo da stimare costi, tempi, vincoli e benefici fin dall'inizio.

\subsubsection{Modello sequenziale (\frgnword{Waterfall model})}
Il modello a cascata (o \frgnword{Waterfall model}) è un modello sequenziale nel quale il processo di realizzazione del software è strutturato in una sequenza strettamente lineare di passi, in cui il ritorno a fasi precedenti non è permesso. Al completamento di ogni passo è prodotta della documentazione, che permette al cliente di analizzare lo stato di avanzamento e approvarlo (\frgnword{document-driven model}). La fase successiva non può iniziare fintanto che quella precedente non è conclusa e approvata.

Questo modello ha origine nell'industria manufatturiera, dove i cambiamenti in corso d'opera hanno costi proibitivi e sono difficilmente attuabili.

Ogni passo del modello a cascata è definito in termini di
\begin{itemize}
	\item Attività e prodotti in \frgnword{input} e \frgnword{output} attesi;
	\item Struttura e contenuto della documentazione;
	\item Responsabilità e ruoli coinvolti;
	\item Scadenze per la consegna della documentazione;
\end{itemize}

Il modello a cascata porta con se dei vantaggi: le pre- e postcondizioni sono ben note e rispettate per definizione, rendendolo facilmente valutabile nei costi e nelle risorse necessarie. Inoltre, per la sua semplicità, è sempre possibile sapere con precisione in che stato di avanzamento si trova il progetto ad un certo istante.

Per contro, questo modello può risultare troppo inflessibile. Inoltre, non producendo prototipi agli stadi intermedi, il cliente riceve il prodotto solo alla fine del processo di sviluppo. È possibile limitare i difetti attuando alcune correzioni che rendano il modello più flessibile, come l'impiego di prototipi usa-e-getta. 

Una scarsa conoscenza delle tecnologie da utilizzare da parte degli \frgnword{stakeholder} potrebbe causare un sostanziale cambiamento dei requisiti in corso d'opera e richiedere l'iterazione. Per questo motivo, è consigliabile impiegare il modello a cascata solo nel caso in cui i requisiti siano ben noti fin dall'inizio e difficilmente modificabili durante lo sviluppo. In casi particolari è possibile permettere iterazioni rieseguendo fasi precedenti, pur snaturando il modello.

%\subsubsection{Modello iterativo}
%\subsubsection{Modello incrementale}
%\subsubsection{Modello evolutivo}
%\subsubsection{Modello a spirale}
%\subsubsection{Modello a componenti}
%\subsubsection{Modelli agili}
%\subsection{Standard di processo}
%\subsubsection{ISO/IEC 12207:1995}

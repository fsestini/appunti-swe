\section{Ciclo di vita del software}
Il ciclo di vita di un prodotto software può essere visto come una macchina a stati, dove ogni stato rappresenta un preciso grado di maturazione del prodotto, e dove ogni transizione tra stati richiede l'esecuzione di attività, raggruppate in processi. Stati e transizioni hanno precise pre- e postcondizioni, determinate da vincoli, regole e strategie dedicate. La durata di uno stato di ciclo di vita o di una transizione tra due è chiamata \strong{fase}. Le fasi mostrano l'avanzamento del progetto in funzione del tempo trascorso.

Ogni prodotto software, lungo il suo ciclo di vita, passa attraverso le fasi di concepimento, sviluppo, utilizzo e ritiro. 

%\subsection{Modelli di ciclo di vita}
%\subsubsection{Modello sequenziale (%\frgnword{Waterfall model})}
%\subsubsection{Modello iterativo}
%\subsubsection{Modello incrementale}
%\subsubsection{Modello evolutivo}
%\subsubsection{Modello a spirale}
%\subsubsection{Modello a componenti}
%\subsubsection{Modelli agili}
%\subsection{Standard di processo}
%\subsubsection{ISO/IEC 12207:1995}

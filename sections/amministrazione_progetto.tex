\section{Amministrazione di progetto}
L'amministrazione di progetto e' associata al ruolo dell'amministratore. L'amministratore di progetto ha il compito di gestire e mantenere organizzato e ben funzionante l'ambiente di lavoro, che comprende:

\begin{itemize}
	\item Regole e procedure, preparate dall'amministratore e approvate dal responsabile di progetto;
	\item Strumenti e servizi di supporto, come l'infrastruttura, i prodotti, i documenti, ecc. \todo{boh, da sistemare}
\end{itemize}

L'amministratore non effettua scelte di gestione, ma mette in pratica scelte tecnologiche concordate con i responsabili di azienda e di progetto;

Con \strong{infrastruttura} si intende un insieme di strumenti che insieme determinano il \frgnword{modus operandi}; con \strong{servizio} si intende un mezzo o strumento che permette all'utilizzatore di raggiungere un obbiettivo senza preoccuparsi dei costi e dei rischi associati: un servizio efficace aumenta l'efficienza complessiva. L'amministratore ha il compito decisivo di gestire i servizi e mantenere l'infrastruttura attiva ed efficiente; tale attivita' e' nascosta, ma continua.

\subsection{Documentazione}
La documentazione associata a un prodotto software ha diverse funzioni:
\begin{itemize}
	\item Agire come un canale di comunicazione tra i membri del team di sviluppo;
	\item Costituire un \frgnword{repository} di informazioni per i manutentori;
	\item Fornire informazioni utili per la pianificazione, il calcolo delle risorse necessarie e la calendarizzazione delle attivita';
	\item Guidare gli utenti all'utilizzo del sistema;
\end{itemize}

I \frgnword{software engineer} sono tipicamente incaricati della produzione della maggior parte della documentazione;

\begin{description}
	\item[Documentazione di processo] Relativa ai processi di sviluppo e manutenzione. Il piano di progetto, il calendario delle attivita', gli standard e i documenti di qualita' ne fanno parte;
	\item[Documentazione di prodotto] Descrive il prodotto, e si divide in documentazione di sistema, usata dagli ingegneri, e documentazione utente;
\end{description}

un documento e' utile solo se:
\begin{enumerate}
	\item e' sempre disponibile e facilmente accessibile;
	\item chiaramente identificato;
	\item corretto nei contenuti;
	\item verificato e approvato;
	\item aggiornato, datato e versionato;
\end{enumerate}

\todo{Inserire roba su information hiding ecc.}

%\subsection{Ambiente di lavoro}
\todo{Scrivere da zero}
%\subsubsection{infrastruttura}
%\subsection{Gestione di configurazione}

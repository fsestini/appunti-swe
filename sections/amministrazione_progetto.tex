\section{Amministrazione di progetto}

L'amministrazione di progetto è l'attività, svolta dall'amministratore, atta ad
equipaggiare, organizzare e gestire l'ambiente di lavoro e di produzione,
secondo regole, procedure, strumenti e servizi a supporto.

L'amministratore \strong{non effettua scelte} di gestione, ma mette in pratica
scelte tecnologiche concordate con i responsabili di azienda e di progetto. Le
sue attività comprendono:

\begin{itemize}
  \item Redazione e manutenzione di regole e procedure, preparate
  	dall'amministratore e approvate dal responsabile di progetto;
  \item Reperimento, organizzazione, gestione e manutenzione delle risorse
  	informatiche per l'erogazione dei servizi di supporto (che includono
  	ambiente, infrastruttura, strumenti, prodotti, documenti.)
\end{itemize}

\begin{description}
  \item[Infrastruttura] insieme di strumenti che determinano il
    \frgnword{modus operandi};
  \item[Servizio] mezzo o strumento che permette all'utilizzatore di raggiungere
    un obiettivo senza preoccuparsi dei costi e dei rischi associati: un
    servizio efficace aumenta l'efficienza complessiva.
\end{description}

L'amministratore ha il compito decisivo di gestire i servizi e mantenere
l'infrastruttura attiva ed efficiente; tale attivita' e' nascosta, ma continua.

\subsection{Documentazione}

La documentazione associata a un prodotto software ha diverse funzioni:

\begin{itemize}
	\item Agisce come un canale di \strong{comunicazione} tra i membri del team di
    sviluppo;
	\item Costituisce un \strong{\frgnword{repository}} di informazioni per i
    mantenitori;
  \item Fornisce informazioni utili alla \strong{pianificazione}, il calcolo
    delle risorse necessarie e la calendarizzazione delle attività;
	\item \strong{Guida} gli utenti all'utilizzo del sistema;
\end{itemize}

\begin{description}
  \item[Documentazione di processo] Relativa ai processi di sviluppo e
    manutenzione. Il piano di progetto, il calendario delle attivita', gli
    standard e i documenti di qualita' ne fanno parte;
  \item[Documentazione di prodotto] Descrive il prodotto, e si divide in
    documentazione di sistema, usata dagli ingegneri, e documentazione utente;
\end{description}

\paragraph{Disponibilità e diffusione}
\label{par:utilita}

Un documento e' utile solo se possiede le seguenti caratteristiche:

\begin{enumerate}
	\item e' sempre disponibile e facilmente accessibile;
	\item chiaramente identificato;
	\item corretto nei contenuti;
	\item verificato e approvato;
	\item aggiornato, datato e versionato;
\end{enumerate}

La \strong{diffusione} della documentazione deve essere strettamente
\strong{controllata}, identificando chiaramente i destinatari. Ogni documento ha
una \strong{lista di distribuzione}; l'amministratore ha il compito di gestire
questa lista e di assicurarne il rispetto.

\subsection{Ambiente di lavoro}

L'ambiente di lavoro rappresenta le persone, i ruoli, le procedure e
l'infrastruttura necessari ai processi di produzione per essere messi in opera.
La sua qualità determina la produttività, e influisce sulla qualità del processo
e del prodotto.

L'ambiente di lavoro deve essere

\begin{itemize}
  \item \strong{Completo}: offrire tutto il necessario per svolgere le attività
    previste;
  \item \strong{Ordinato}: facile trovare ciò che si cerca;
  \item \strong{Aggiornato}: il materiale obsoleto non deve causare intralcio.
\end{itemize}

\subsubsection{Infrastruttura}

L'infrastruttura e' una serie di elementi strutturali interconnessi, servizi e
strumenti che insieme forniscono un \frgnword{framework} a supporto delle
operazioni e dei processi. Comprende risorse hardware (server, workstation,
ecc.) e software (IDE, controllo di versione, ecc.). I servizi offerti
dall'infrastruttura sono sotto la responsabilita' dell'amministratore.

\subsection{Gestione di configurazione}

Un prodotto software e' composto da diverse parti separate, unite secondo delle
regole rigorose che costituiscono la \strong{configurazione}. L'attività a
gestione delle regole di configurazione è chiamata
\strong{gestione di configurazione}, e deve \strong{automatizzata} con strumenti
adatti.

\paragraph{\frgnword{Configuration item}}
\label{par:configuration_item}

Tutto ciò che è posto sotto controllo di configurazione è definito
\strong{\frgnword{configuration item}}. Ogni \frgnword{configuration item} ha
un'identità unica (nome, data, autore, registro delle modifiche, stato
corrente), e si trova spesso in piu'versioni.

La gestione di configurazione identifica e controlla i configuration item,
definendo \strong{quali} compongono il prodotto, e \strong{come} questi sono
aggregati nel processo di \strong{build}.

\paragraph{Baseline}
\label{par:baseline}

La gestione di configurazione identifica e controlla le \frgnword{baseline},
collezione di \frgnword{configuration item} formalmente approvati, che
costituiscono un sistema. Costituisce una descrizione degli attributi di un
prodotto ad un certo stato di avanzamento, su cui è possibile fare una
valutazione di \strong{progresso}. È tipicamente associata ad una
\strong{milestone}.

Le attivita' che compongono il processo di gestione di configurazione sono le
seguenti:

\begin{itemize}
	\item Identificazione della configurazione;
	\item Gestione dei cambiamenti;
	\item Controllo di versione;
	\item Processo di \frgnword{build};
	\item \frgnword{Release management};
\end{itemize}

\paragraph{Identificazione della configurazione}

L'identificazione della configurazione si occupa di impostare e mantenere le
\frgnword{baseline}. Stabilisce e mantiene in maniera incrementale i
\frgnword{configuration item} durante tutti il loro ciclo di vita. L'esistenza
di \frgnword{baseline} ben definite permette riproducibilita', tracciabilita' e
analisi del processo di sviluppo.

\paragraph{Gestione dei cambiamenti}

Il processo di gestione dei cambiamenti analizza i costi e i benefici relativi
alle proposte di modifica ricevute, approva quelle significative e tiene traccia
di quali componenti del sistema vengono modificati.

Le proposte di cambiamento possono provenire da:

\begin{itemize}
	\item Utenti (bug report);
	\item Sviluppatori;
	\item Competizione con altre aziende;
\end{itemize}

Le proposte di modifica seguono un severo processo di analisi, decisione,
implementazione e verifica. Ogni richiesta di modifica deve essere presentata
tramite un \frgnword{change request form} (CRF), nel quale vengono inoltre
memorizzate decisioni e raccomandazioni riguardanti la modifica, il costo
stimato e le date di richiesta, approvazioine, implementazione e validazione.
Grazie a strumenti di \frgnword{issue tracking} (o \frgnword{ticketing}) si e'
in grado di tenere traccia dello stato di ogni richiesta di modifica.

\paragraph{Controllo di versione}

Ogni componente software si trova in piu' \strong{versioni}, istanze
funzionalmente distinte dello stesso componente, alcune di esse destinate al
rilascio all'utente. Il processo di controllo di versione si occupa di tenere
traccia delle differenti versioni di ogni componente del sistema, e fa in modo
che il lavoro di ogni sviluppatore non interferisca con quello degli altri.

Il controllo di versione riguarda sostanzialmente la gestione di
\frgnword{codeline} e \frgnword{baseline}. Una \frgnword{codeline} e' una
sequenza di versioni del sorgente di un componente, dove ogni versione deriva da
quella precedente. Il processo si serve di un \frgnword{repository}, una sorta
di \frgnword{database}/\frgnword{filesystem} nel quale vengono memorizzati i
\frgnword{configuration item} in tutte le loro versioni.

\subsection{Norme di progetto}
\label{sub:norme_di_progetto}

Le norme di progetto costituiscono linee guida per le attività di progetto,
riguardo

\begin{itemize}
  \item Organizzazione e convenzioni sugli strumenti di sviluppo;
  \item Organizzazione della comunicazione e cooperazione;
  \item Attività di codifica;
  \item Gestione dei cambiamenti.
\end{itemize}

Possono essere \strong{regole}, verso il quale è richiesto e accertato il
rispetto, o \strong{raccomandazioni}, prassi desiderabile senza verifica di
rispetto. Il contesto definisce la portata della norma: troppe regole sono di
difficile attuazione e verifica.

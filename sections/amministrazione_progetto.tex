\section{Amministrazione di progetto}
L'amministrazione di progetto e' associata al ruolo dell'amministratore. L'amministratore di progetto ha il compito di gestire e mantenere organizzato e ben funzionante l'ambiente di lavoro, che comprende:

\begin{itemize}
	\item Regole e procedure, preparate dall'amministratore e approvate dal responsabile di progetto;
	\item Strumenti e servizi di supporto, come l'infrastruttura, i prodotti, i documenti, ecc. \todo{boh, da sistemare}
\end{itemize}

L'amministratore non effettua scelte di gestione, ma mette in pratica scelte tecnologiche concordate con i responsabili di azienda e di progetto;

Con \strong{infrastruttura} si intende un insieme di strumenti che insieme determinano il \frgnword{modus operandi}; con \strong{servizio} si intende un mezzo o strumento che permette all'utilizzatore di raggiungere un obbiettivo senza preoccuparsi dei costi e dei rischi associati: un servizio efficace aumenta l'efficienza complessiva. L'amministratore ha il compito decisivo di gestire i servizi e mantenere l'infrastruttura attiva ed efficiente; tale attivita' e' nascosta, ma continua.

%\subsection{Documentazione}
%\subsection{Ambiente di lavoro}
\todo{Scrivere da zero}
%\subsubsection{infrastruttura}
%\subsection{Gestione di configurazione}

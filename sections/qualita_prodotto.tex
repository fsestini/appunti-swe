\section{Qualità di prodotto}
Con \strong{qualità} si intende l'insieme delle caratteristiche di un'entità, sia essa un prodotto, un processo o un'intera organizzazione, che ne determinano la capacità di soffisfare esigenze espresse e implicite. Le attività di gestione della qualità costituiscono il \strong{sistema qualità}, con cui si identifica la struttura organizzativa, le responsabilità, le procedure, i procedimenti e le risorse messe in atto per il perseguimento della qualità. Tali attività  si occupano di stabilire un \frgnword{framework} di processi e standard che assicurino un alto livello di qualità, e applicare specifici processi di qualità atti a verificare la corretta aderenza al \frgnword{framework} scelto e la conformità del prodotto finale agli standard.

Vi sono diverse visioni, o punti di vista, da cui valutare la qualità, e su cui il sistema qualità agisce:
\begin{itemize}
	\item Intrinseca: conformità ai requisiti, idoneità all'uso;
	\item Relativa: soddisfazione del cliente;
	\item Quantitativa: misura del livello di qualità per confronto;
\end{itemize}

\paragraph{Pianificazione di qualità}
Le attività del sistema qualità mirate a fissare gli obbiettivi di qualità, i processi e le risorse necessarie per conseguirli costituiscono la \strong{pianificazione di qualità}. Essa è prerequisito per la gestione della qualità e per il miglioramento continuo\footnote{Vedi ciclo di Deming}, e fissa strategie e politiche sia a livello aziendale che di singolo progetto. La pianificazione della qualità è parte integrante dei processi di sviluppo \frgnword{plan-driven}, mentre viene affrontata in modo meno formale nei modelli agili.

Il processo di pianificazione della qualità sviluppa un piano di qualità per il progetto, che espone le caratteristiche di qualità del software desiderate, e descrive come esse debbano essere valutate. 

\paragraph{Controllo di qualità}
Con il termine \strong{controllo di qualità} si indicano le attività del sistema qualità pianificate e attuate affinchè il prodotto soddisfi i requisiti attesi. Comprende le attività di \strong{\frgnword{quality assurance}} (QA), ovvero la definizione di processi e standard che dovrebbero portare a prodotti di alta qualità, e l'introduzione di processi di qualità nel processo di sviluppo (controllo preventivo della qualità).

Include le attività di verifica e validazione e i processi incaricati di controllare che le procedure di qualità siano correttamente applicate.

\subsection{Standard di qualità}
Gli standard giocano un ruolo molto importante nella gestione della qualità del software. Essi:
\begin{itemize}
	\item Catturano e rappresentano le conoscenze, l'esperienza e le \frgnword{best-practice} dell'azienda;
	\item Forniscono un \frgnword{framework} per l'attuazione di \frgnword{quality assurance};
	\item Supportano la continuità nel momento in cui il lavoro svolto da una persona viene continuato da un'altra: processi standardizzati sono facilmente comprensibili da nuovi assunti;
\end{itemize}

Un uso errato degli standard può avere effetti negativi. E' importante che le norme siano snelle, chiaramente complensibili e supportabili da strumenti automatizzati.
\begin{itemize}
	\item Se gli standard sono poco comprensibili, il personale può percepirli come irrilevanti o bloccanti;
	\item La loro attuazione cieca può comportare eccessi di burocrazia;
	\item Senza il supporto di strumenti informatici, possono richiedere tediose attività manuali;
\end{itemize}


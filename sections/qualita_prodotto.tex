\section{Qualità di prodotto}
Con \strong{qualità} si intende l'insieme delle caratteristiche di un'entità, sia essa un prodotto, un processo o un'intera organizzazione, che ne determinano la capacità di soffisfare esigenze espresse e implicite. Le attività di gestione della qualità costituiscono il \strong{sistema qualità}, con cui si identifica la struttura organizzativa, le responsabilità, le procedure, i procedimenti e le risorse messe in atto per il perseguimento della qualità. Tali attività  si occupano di stabilire un \frgnword{framework} di processi e standard che assicurino un alto livello di qualità, e applicare specifici processi di qualità atti a verificare la corretta aderenza al \frgnword{framework} scelto e la conformità del prodotto finale agli standard.

Vi sono diverse visioni, o punti di vista, da cui valutare la qualità, e su cui il sistema qualità agisce:
\begin{itemize}
	\item Intrinseca: conformità ai requisiti, idoneità all'uso;
	\item Relativa: soddisfazione del cliente;
	\item Quantitativa: misura del livello di qualità per confronto;
\end{itemize}

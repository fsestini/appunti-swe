\section{Qualità di processo}

Per ottenere qualità, è necessario

\begin{itemize}
  \item Definire il processo, per poterlo facilmente controllare;
  \item Controllare il processo, per migliorarne efficienza ed efficacia;
  \item Usare buoni strumenti di valutazione.
\end{itemize}

\paragraph{Gestione della Qualità (QM)}
\label{par:gestione_della_qualit_}

La Gestione della Qualità assicura la consistenza di un'azienda, un prodotto o
un servizio. Ha quattro principali componenti:

\begin{itemize}
  \item Pianificazione della qualità;
  \item Controllo della qualità;
  \item Quality assurance;
  \item Miglioramento della qualità.
\end{itemize}

\paragraph{Sistema di Gestione della Qualità (SGQ)}
\label{par:sistema_di_gestione_della_qualit_}

Un Sistema di Gestione della Qualità (SGQ) è una \strong{collezione} di
\strong{processi} aziendali focalizzati al raggiungimento di obiettivi e
politiche di qualità per soddisfare i bisogni del cliente. Rappresenta la
struttura organizzativa, le politiche, le procedure, i processi e le risorse
necessarie a realizzare la Gestione della Qualità (Quality Management).

\subsection{ISO 9000}
\label{sub:iso_9000}

ISO 9000 è uno standard riguardante la gestione della qualità. Il suo obbiettivo
è di integrare un sistema di qualità in un'azienda, migliorando la produttività,
riducendo i costi non necessari e assicurando qualità di prodotto e di processo.

ISO 9000 rappresenta una famiglia di standard:

\begin{itemize}
  \item ISO 9000:2005: Fondamenti e glossario. Modelli di qualità neutri
    rispetto al dominio di applicazione;
  \item ISO 9001:2000: Sistema Gestione Qualità. ISO 9000 calata nei processi
    produttivi;
  \item ISO 90003:2004: Software engineering. Linee guida per l'applicazione di
    ISO 9001 al software.
\end{itemize}

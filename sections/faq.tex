\section{\foreignword{Frequently asked questions}}

\subsection{Progettazione software}

\begin{question}
  Fornire una definizione del concetto di ``architettura software'' radicata nel
  dominio dell'insegnaemto. A partire da essa, definire i concetti di "design
  pattern" e ``framework'', mettendoli in relazione tra loro e con la
  definizione di ``architettura''. Indicare infine dove e quando tali tra concetti
  svolgano un ruolo significativo all'interno del ``processo di sviluppo''.
\end{question}

L'architettura software descrive l'organizzazione di un sistema, la
decomposizione in componenti e le relazioni e i vincoli tra esse.

Il design pattern rappresenta una soluzione architetturale come
\foreinword{best practice} alla risoluzione di un problema noto.

Un pattern architetturale non descrive l'architettura del sistema, ma ne
definisce il modello, il tipo di componenti, le loro interazioni i vincoli su
esse.

Un framework è un insieme integrato di componenti software prefabbricate.
Costituisce la base per la progettazione \foreinword{bottom-up} di
un'architettura, e impone in maniera top-down uno stile architetturale.

Questi concetti hanno un ruolo significativo nel processo di sviluppo, in
particolare durante le attività di progettazione architetturale e di dettaglio.
La buona scelta di framework facilità il riuso del codice e migliora
efficienza ed efficacia dell'attività di codifica.

Una buona architettura ha inoltre impatto sulle attività di testing e verifica
della qualità.

\section{\foreignword{Frequently asked questions}}

\subsection{Gestione di progetto}

\begin{question}
  Discutere le differenze informative tra il ``diagramma di PERT'' e il
  ``diagramma di Gantt''.
\end{question}

Il diagramma di Gantt offre una dislocazione temporale delle attività di
progetto, offrendo informazioni riguardo l'inizio, la fine e la durata delle
attività, oltre alla loro sovrapposizione. Permette, inoltre, di confrontare la
pianificazione con il lavoro effettivamente svolto.

Il diagramma PERT fornisce prevalentemente informazioni riguardo la
dipendenza temporale tra le attività, e permette di ragionare sulle scadenze di
un progetto. PERT non focalizza l'attenzione sulla disposizione temporale delle
singole attività o sul tempo impiegato da esse, ma bensì sulla relazione che
queste attività hanno tra loro, e su come il loro svolgimento si riflette sulle
altre e sull'intero progetto.

\subsection{Progettazione software}

\begin{question}
  Fornire una definizione del concetto di ``architettura software'' radicata nel
  dominio dell'insegnaemto. A partire da essa, definire i concetti di "design
  pattern" e ``framework'', mettendoli in relazione tra loro e con la
  definizione di ``architettura''. Indicare infine dove e quando tali tra concetti
  svolgano un ruolo significativo all'interno del ``processo di sviluppo''.
\end{question}

L'architettura software descrive l'organizzazione di un sistema, la
decomposizione in componenti e le relazioni e i vincoli tra esse.

Il design pattern rappresenta una soluzione architetturale come
\foreinword{best practice} alla risoluzione di un problema noto.

Un pattern architetturale non descrive l'architettura del sistema, ma ne
definisce il modello, il tipo di componenti, le loro interazioni i vincoli su
esse.

Un framework è un insieme integrato di componenti software prefabbricate.
Costituisce la base per la progettazione \foreinword{bottom-up} di
un'architettura, e impone in maniera top-down uno stile architetturale.

Questi concetti hanno un ruolo significativo nel processo di sviluppo, in
particolare durante le attività di progettazione architetturale e di dettaglio.
La buona scelta di framework facilità il riuso del codice e migliora
efficienza ed efficacia dell'attività di codifica.

Una buona architettura ha inoltre impatto sulle attività di testing e verifica
della qualità.

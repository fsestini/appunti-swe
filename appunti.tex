\documentclass[10pt,a4paper]{article}
\usepackage[italian]{babel}
\usepackage[utf8]{inputenc}
\usepackage{amsmath}
\usepackage{amsfonts}
\usepackage{amssymb}
\usepackage{a4wide}
\usepackage{graphicx}
\usepackage{todonotes}
\usepackage[toc]{glossaries}

\makeglossaries

%\newglossaryentry{Ingegneria del Software}
{
	name={Ingegneria del Software},
	description={Approccio sistematico, disciplinato e quantificabile allo sviluppo, il mantenimento e il ritiro del software}
}

\newglossaryentry{best practice}
{
	name={best practice},
	description={Metodo di lavoro che l'esperienza e lo studio hanno provato avere i migliori risultati in circostanze specifiche e ben note}
}

\newglossaryentry{efficienza}
{
	name=efficienza,
	description={Valore indicativo inversamente proporzionale alla quantità di risorse consumate nell'esecuzione di un processo di produzione}
}

\newglossaryentry{efficacia}
{
	name=efficacia,
	description={Valore indicativo di quanto un prodotto o un processo soddisfano i requisiti di conformità e qualità}
}

\newglossaryentry{ciclo di vita}
{
	name={ciclo di vita},
	description={Insieme di tutti gli stati assunti dal prodotto dal concepimento al ritiro, e dei processi che lo portano da uno stato all'altro}
}

\newglossaryentry{stakeholder}
{
	name=stakeholder,
	description={Soggetto (o un gruppo di soggetti) influente nei confronti di un'iniziativa economica, sia essa un'azienda o un singolo progetto}
}

\newglossaryentry{processo}
{
	name={processo},
	description={Struttura metodologica, normativa e strategica che caratterizza, suddivide e ordina le attività di progetto}
}

\newglossaryentry{milestone}
{
	name={milestone},
	description={Stato del progetto in cui e' possibile accertare un progresso significativo}
}

\newglossaryentry{metrica}
{
	name=metrica,
	description={Sistema di misurazione}
}

\newcommand{\strong}[1]{\textbf{#1}}
\newcommand{\frgnword}[1]{\textit{#1}}

\title{Ingegneria del Software}
\author{Filippo Sestini}

\begin{document}
\maketitle
\tableofcontents

\section{Introduzione}
\subsection{Cos'è l'ingegneria del software}
L'Ingegneria è l'applicazione di principi matematici e fisici per fini pratici. Tali fini sono spesso di interesse sociale e civile, non solo legati a prodotti di consumo. \strong{L'Ingegneria del Software} è l'approccio sistematico, disciplinato e quantificabile allo sviluppo, il mantenimento e il ritiro del software. È sistematico e disciplinato in quanto fatto secondo un piano prefissato o un insieme di regole in modo metodico e standardizzato. È quantificabile perchè si vuole che il costo di un processo, in termini di tempo e denaro in particolare, sia noto a priori.

L'ingegnere del software segue una \frgnword{best practice}, un metodo di lavoro che l'esperienza e lo studio hanno provato avere i migliori risultati in circostanze specifiche e ben note.

L'Ingegneria del Software è strettamente correlata alle seguenti discipline:
\begin{itemize}
	\item Informatica: l'Ingegneria del Software non è una branca dell'Informatica, ma una disciplina ingegneristica che si appoggia in parte su essa;
	\item Matematica: analisi numerica, matematica discreta, ricerca operativa, statistica, ecc.;
	\item Economia e management;
	\item Psicologia e sociologia;
\end{itemize}

L'ingegnere del software deve assicurare la qualità del prodotto richiesta dai requisiti, massimizzando l'\strong{efficacia} dei processi, contenendo i costi e i tempi di produzione e minimizzando l'uso di risorse, massimizzandone quindi l'\strong{efficienza}. I valori di efficienza ed efficacia sono inversamente proporzionali: la migliore soluzione al problema (tipicamente ve n'è più di una) tenderà ad avere valori ottimi per entrambi.

Vi sono più tipi di prodotti software: prodotti generici, \frgnword{stand-alone}, prodotti da aziende di sviluppo software e messi apertamente sul mercato, e prodotti specifici, commissionati da un particolare cliente.

Ogni prodotto software ha un ciclo di vita, che rappresenta tutti gli stati assunti dal prodotto dal concepimento al ritiro. Durante il loro ciclo di vita, molti sistemi software sono soggetti a varie forme di \strong{manutenzione}, che può essere di tre tipi:

\begin{itemize}
	\item Correttiva: corregge eventuali difetti riscontrati;
	\item Adattiva: modifica il software per riflettere eventuali richieste del cliente o variazioni nel mercato;
	\item Evolutiva: aggiunge nuove funzionalità al sistema;
\end{itemize}

La mantenibilità è una qualità essenziale del software.

%\subsection{Il \frgnword{software engineer}}

%\subsection{Le persone}
%\subsection{Il progetto}
%\subsection{Il processo}
%\section{Processi software}
%\subsection{Processi, modelli, progetti}
%\subsection{Principio di miglioramento continuo}
%\section{Ciclo di vita del software}
%\subsection{Modelli di ciclo di vita}
%\subsubsection{Modello sequenziale (%\frgnword{Waterfall model})}
%\subsubsection{Modello iterativo}
%\subsubsection{Modello incrementale}
%\subsubsection{Modello evolutivo}
%\subsubsection{Modello a spirale}
%\subsubsection{Modello a componenti}
%\subsubsection{Modelli agili}
%\subsection{Standard di processo}
%\subsubsection{ISO/IEC 12207:1995}
%\section{Gestione di progetto}
%\subsection{Gestione dei rischi}
%\subsection{Ruoli}
%\subsubsection{Analisti e progettisti}
%\subsubsection{Programmatori e verificatori}
%\subsubsection{Responsabile di progetto}
%\subsubsection{Amministratore}
%\subsection{Pianificazione di progetto}
%\subsubsection{Project scheduling}
%\subsubsection{Allocazione delle risorse}
%\subsubsection{Stima dei costi di progetto}
%\subsection{Piano di progetto}
%\section{Amministrazione di progetto}
%\subsection{Documentazione}
%\subsection{Ambiente di lavoro}
%\subsubsection{infrastruttura}
%\subsection{Gestione di configurazione}

%\printglossaries

\end{document}